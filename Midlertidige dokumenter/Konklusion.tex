I dette projekt blev der ønsket undersøgt om automatiserede hjem kunne forbedres, og hvordan dette kunne lade sig gøre. For at opnå dette blev en dataindsamling gennemført, med henblik på at finde dokumentation på hvilke problemer der måtte findes inden for emnet. Ud fra denne dataindsamling blev det konstateret, at der fandtes et hul i styringen af automatiserede hjem, da disse ikke kan styres af stemmer. Vi ser en mulighed for at effektivisere og bekvemeliggøre dagligdagen for beboere i automatiserede huse, ved at udvikle et stemmestyringssystem af automatiserede hjem, som kan modtage stemmeinput fra en bruger og udføre handlinger eller scenarier specificeret af brugeren.\\
Da gruppen er software orienteret blev produktet afgrænset til at behandle stemmeinput efter stemme-til-tekst processen var foregået, dog blev forskellige idéer omkring hardware forudsætninger udtænkt, blandt andet stikkontaker og lampeudtag med indbyggede mikrofoner, således at stemmeinput, samt gensvar fra stemmestyringssystemet, altid ville kunne modtages, hvorend brugeren er i et hus. Et eksempel på placeringer af disse i et hjem tilhørende en standard familie på fire blev ligeledes udarbejdet.\\
Det blev undersøgt hvilken viden der var nødvendig for at lave produktet. Det blev fastslået at en tilstandsmaskine var nødvendig at inkludere i det endelige produkt, da programmet skal igennem en mængde seperate faser, som alle kan lede ud i forskellige handlinger. Det blev ligeledes konkluderet, at en stavekontrol var nødvendig, da forvrænget stemmeindput, resulterende i manglende eller ændrede bogstaver, er et aktuelt problem ved stemmestyring.\\
Ud fra denne viden er der blevet konstrueret et C produkt, som kan styre et hjem, ved at tage simuleret stemme indput i form af tekst, kontrollere at det er brugerens mente indput, ved hjælp af stavekontrol og simulerer udførelsen af handlingen eller scenariet specificeret af brugeren, resulterende i forøget komfort og tidsbesparelse, idet at hjemmets elektroniske applikationer kan styres let, hvorend brugeren befinder sig i hjemmet. Med denne løsning anses problemformuleringen besvaret.