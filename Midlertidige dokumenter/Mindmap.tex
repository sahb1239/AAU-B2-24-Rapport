Før vi overhovedet har kunnet begynde vores arbejde, er det relevant at undersøge, hvilke begreber, der hersker indenfor vores emne, hvem der har interesse for emnet og om der er nogle problemer, som skal løses indenfor emnet.\\
For at overskuliggøre vores emne "Automatisering af huse", har vi været nødt til at skabe noget struktur. \\

Først har vi lavet noget simpel litteratur søgning for at finde inspiration til emnet, og ud fra denne har vi lavet et mindmap, for at få nogle idéer til, hvilke eksisterende teknologier der er, hvem der har interesse i emnet og hvilke problemer, der evt. måtte herske indenfor emnet. Dette skaber struktur og giver os noget konkret at arbejde ud fra.

Nedenfor ses mindmappet (figur \ref{fig:Coggle}) som vi lavede i starten af projektet. \\\\
\figurw{Figurer/BrainstormCoggle.png}{Mindmap}{Coggle}{1.0}

Mindmappet skabte et udgangspunkt for, at vi kunne undersøge, hvilke problemer der måtte herske omkring emne og ud fra dette komme frem til en problemformulering. \\Før vi kan lave en problemformulering er vi dog nødt til at vide, om der overhovedet hersker et problem. Vi besluttede derfor at lave et spørgeskema for at undersøge, om der er nogle problemer, der kunne være relevante at belyse indenfor vores emne, "Automatisering af huse".\\