\section{Forord}
Til problemanalysen er det blevet valgt at fremstille et spørgeskema. Formålet ved dette spørgeskema var at høre eksisterende ejere af automatiserede huse om interessen i et mere inteligent hus der forudsige at du eks. ønsker at der skal laves kaffe, eller at lyset langsomt tænder efter at en film er færdig på fjernsynet.\\
Denne form for automatisering med kunstig inteligens kan hele tiden udvides og der vil altid være noget huset kan lære om brugeren\\
Spørgeskemaet har været åben for besvarelser i 5 dage, i perioden er der indsamlet 272 fuldendte besvarelser, mens der er registreret 40 ufuldendte besvarelser.\\For at få flere besvarelser er linket til spørgeskemaet blevet lagt ind på flere automatiseringsfora, derudover har alle gruppemedlemmer lagt spørgeskemaet op på deres Facebook profil.\\

\subsection{Målgrupper}
For at få nogle releavante data'er er det også vigtigt at kunne finde den primære målgruppe som man forventer kunne have interesse i et automatiseret hus\\\\
\indent Personer der ejer et hus, lejlighed eller lignende\\
\indent Lønmodtagere i den offentlige eller private sektor\\\\
Denne profil er valgt fordi at det ikke forventes at folk der ikke ejer et hus, lejlighed eller lignende vil bruge penge på at gøre udlejers lejlighed i bedre stand. Derudover forventes det heller ikke at denne gruppe har råd til det, da de ikke er lønmodtagere.

\subsection{Spørgeskemaets opbygning}
Spørgeskemaet er bygget op med SurveyXact der er en spørgeskematjeneste udviklet af Ramboll Managment Consulting A/S. Tjenesten er valgt pga. det er en tjeneste der bliver stillet til rådighed af Aalborg Universitet, derudover er der også mange funktioner der normalt kun fås i betalingsudgaver af spørgeskematjenester (et eksempel er Surveymonkey \cite{surveymonkeyprice}).\\Nedenfor gennemgås alle spørgsmål og hvorfor at de var relavante at have med.
\\\\