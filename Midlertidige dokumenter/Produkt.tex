% Herunder præsenteres vores slut produkt. Det program vi er nået frem til.
Produkt















\subsubsection{Stavekontrol} %Martin. Skal flyttes over i implementation sikkert.

Som stavekontrol bruges redigeringsafstand metoden. Denne løsning er hovedsageligt baseret på Peter Norvigs (Chef af Googles forsknings afdeling) løsning.\cite{EditDistance}\\\\

Følgende handlingsrækkefølge sker når stavekontrollen startes og et ord indføres. "Input" vil herefter referere til det i stavekontrollen indførte ord. 

\begin{enumerate}
  \item Input kontrolleres imod ord fra en database af mulige korrekte ord. Hvis input matcher en af disse ord vil stavekontrollen afslutte uden yderligere handlinger.
  \item Hvis input ikke matcher et ord i databasen udføres følgende redigeringer af input (forklaret i teori):
  \begin{enumerate}
    \item Indsætning.
    \item Sletning.
    \item Erstatning.
    \item Ombytning.
  \end{enumerate}
  \item Redigerings afstanden er nu en. bogstavskombinationerne kontrolleres imod ordene i databasen. Hvis der kommer udslag spørger programmet brugeren om det fundne ord var det han/hun mente. Hvis det var benyttes dette til at fuldføre kommandoen.
  \item Hvis det ikke var fortsætter stavekontrollen med endnu en redigering, nu på de bogstavskombinationer som før blev fundet.
  \item Redigerings afstanden er nu to. Samme procedure som ved første redigering sker. Hvis der ikke kommer udslag i sammenligningen af bogstavskombinationer og database ord informeres brugeren om at input ikke kunne forstås.
\end{enumerate}

Der udføres ikke flere redigeringer efter redigeringsafstand to da antallet af komputationer\mafix{cite til teori afgrænsning} for hvert ord bliver for store derefter.

% Mulige underafsnit: Main, som skal indeholde vores main og forklare hvad denne gør. Input, hvor funktioner som håndterer input forklares, måske database herunder. Output: Output funktioner forklares. Test: Test af programmet.