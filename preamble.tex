% Bemærk ændret til at passe til processanalysen!!!

\documentclass[a4paper,11pt]{report}    % report tilføjer chapter
\usepackage{fourier}                    % Font
\usepackage[utf8]{inputenc}             % Encoding
\usepackage[danish]{babel}              % Dansk sprog
\usepackage{a4}                         % A4 format
\usepackage[margin=3.0cm]{geometry}     % Til at sætte margin
\usepackage{fancyhdr}                   % pæn header og footer
\usepackage{lastpage}                   % til at finde sidste side
\usepackage{hyperref}                   % Referencer i PDF dokumentet og referencer til labels
\usepackage{float}                      % Til at sætte figurer korrekt
\usepackage{url}                        % Til at lave URL
\usepackage[final,danish]{fixme}        % Fixme final/draft
\usepackage{titlesec}                   % Til at ændre på størrelsen på sections
\usepackage{xr}                         % Gør det muligt at cross referere i andre filer
\usepackage{amsmath}                    % Matematik-ting
\usepackage{mathtools}                  % Flere matematik-ting
\usepackage{multirow}                   % Multirows i tabel
\usepackage{threeparttable}             % Til fodnoter i tabeller
\usepackage{listingsutf8}               % Til sourcecode highlight i utf8 format
\usepackage{natbib}                     % Udvidelse af kilder
\usepackage[toc,page]{appendix}         % Til bilag
\usepackage{enumitem}

\let\tnote\relax                        % Fjerner ctable konflikt men ødelægger potientielt noget andet

\usepackage{graphicx}                   % Billeder
\usepackage{pdfpages}                   % PDF sider
\usepackage{color}                      % Farver

% Dokument informationer
\newcommand{\gruppen}{B2-24}
\newcommand{\rapportnavn}{Automatisér dit hjem - Procesanalyse}
\newcommand{\currentpage}{\thepage}     % nuværende side
\newcommand{\numpages}{\pageref{LastPage}}% antal sider
\newcommand{\sidetal}{Side {\currentpage} af {\numpages}} % side informationer
\newcommand{\pagetitle}{\rightmark}     % nuværende sektion
\newcommand{\gruppenummer}{B2-24}
\newcommand{\figuregroup}{(Figuren er fremstillet af gruppen)}
\newcommand{\tabelgroup}{(Tabellen er fremstillet af gruppen)}
\newcommand{\authjem}{Automatiseret hjem}

% Definerer titel osv.
\title{\rapportnavn}
\author{\gruppen}
\date{December 2013}

% Setup af header og footer
\fancypagestyle{plain}{                 % plain er standard for kapitler
    \fancyhead[RO]{Automatisér dit hjem}
    \fancyhead[LO]{Procesanalyse}
    \fancyfoot[CO]{\sidetal}
}
\pagestyle{plain}

% Fjerner afstand mellem kapitel og sætter størrelsen på skriften - se evt. her: http://tex.stackexchange.com/questions/63390/how-to-decrease-spacing-before-chapter-title
\titleformat{\chapter}[display]{\normalfont\huge\bfseries}{\chaptertitlename\ \thechapter}{20pt}{\Huge}
\titlespacing*{\chapter}{0pt}{0pt}{20pt}

% Skriftstørrelse på sections
\titleformat*{\section}{\LARGE\bfseries}
\titleformat*{\subsection}{\Large\bfseries}
\titleformat*{\subsubsection}{\large\bfseries}
\titleformat*{\paragraph}{\normalsize\bfseries}
\titleformat*{\subparagraph}{\normalsize\bfseries}

% Opretter subsubsubsection og subsubsubsubsection
\newcommand{\subsubsubsection}[1]{\paragraph{#1}\mbox{}\\}
\newcommand{\subsubsubsubsection}[1]{\subparagraph{#1}\mbox{}\\}

% Definerer antallet af overskrifter der har tal, samt antallet af overskrifter i indholdsfortegnelsen
\setcounter{secnumdepth}{1}             % 5 Fix for paragraph er en overskrift
\setcounter{tocdepth}{1}                % 5 Fix for paragraph er i indholdsfortegnelse

% Indent til og fra
\newcommand{\enableIndent}{\setlength\parindent{3pt}}
\newcommand{\disableIndent}{\setlength\parindent{0pt}}
\disableIndent

\hyphenation{}                         % Orddeling

% Ændrer indstillingerne for caption for figurer, tabeller og kodestykker - tilføjer bla. kusiv og gør typenavnet bold (eks. Figur 1.1: bliver fed)
\usepackage[font=small,format=plain,labelfont=bf,up,textfont=it,up]{caption}

% Danske bogstaver i kodefiler
\lstset{literate=%
    {æ}{{\ae}}1
    {å}{{\aa}}1
    {ø}{{\o}}1
    {Æ}{{\AE}}1
    {Å}{{\AA}}1
    {Ø}{{\O}}1
}

% Source code farve
\definecolor{gray95}{gray}{.95}

% Source code
\lstdefinestyle{sourceC}
{ 
	numbers=left,
	numbersep=5pt,
	stepnumber=1,
	captionpos=b,
	keywordstyle=\color[rgb]{0,0,1},
	commentstyle=\color[rgb]{0.133,0.545,0.133},
	stringstyle=\color[rgb]{0.627,0.126,0.941},
	backgroundcolor=\color{gray95},
	frame=lrtb,
	framerule=0.5pt,
	linewidth=1.00\textwidth,
	tabsize=4,
	numberbychapter=true,
	basicstyle=\ttfamily\footnotesize,
	language=C,
	breaklines=true,
	showstringspaces=false,
	breakindent=20pt
}

% Source code tekster
\renewcommand{\lstlistingname}{Kode}
\renewcommand{\lstlistlistingname}{Koder}

% Kildeliste
\bibpunct[, ]{[}{]}{;}{n}{,}{,} 		% Definerer de 6 parametre ved Harvard henvisning (bl.a. parantestype og seperatortegn)
%\bibliographystyle{Bibtex/Vancouver}    % Udseende af litteraturlisten
\bibliographystyle{unsrtnat} %plainnat}            % Inkluderer bl.a. isbn numre

% Oprettelse af links inde i pdf dokumentet
\hypersetup{
    pdftitle={\rapportnavn},
    pdfauthor={\gruppen},
    pdfsubject={\rapportnavn},
    bookmarksnumbered=true,
    bookmarksopen=false,
    bookmarksopenlevel=1,
    colorlinks=false,
    pdfstartview=Fit,
    pdfpagemode=UseOutlines,
    pdfpagelayout=TwoPageRight,
    pdfborder = {0 0 0}
}

\hbadness=10001          % Slår alle underfull badness warnings fra
\hfuzz=5.002pt           % Slår de ligegyldige overfull warnings fra

% FixMe
% http://mirrors.dotsrc.org/ctan/macros/latex/contrib/fixme/fixme.pdf
\newcommand{\fix}[1]{\fxnote{#1}}
\newcommand{\dfix}[1]{\fxfatal{#1}}
\renewcommand\fxdanishfatalname{Fejl}
\renewcommand\fxdanishnotename{Note}
\renewcommand{\fixme}[1]{\fix{Benyt ikke fixme}}

% Figurer
% Figur med standard størrelse
\newcommand{\figur}[3]{
		\begin{figure}[H] \centering \em
			\includegraphics{#1}
			\caption{#2}\label{fig:#3}
		\end{figure} 
}

% Figur med mulighed for at vælge bredde
\newcommand{\figurw}[4]{
		\begin{figure}[H] \centering \em
			\includegraphics[width=#4\textwidth]{#1}
			\caption{#2}\label{fig:#3}
		\end{figure} 
}

% Figur med størrelsen 1.0
\newcommand{\figurf}[3]{
		\begin{figure}[H] \centering \em
			\includegraphics[width=1.0\textwidth]{#1}
			\caption{#2}\label{fig:#3}
		\end{figure} 
}

% Kode kommandoer
% Indlæs alt C-kode i fil
\newcommand{\kode}[3]{
            \lstinputlisting[style=sourceC, language=C, caption={#2}, label={code:#3}]{#1}
}

% Indlæs alt C-kode mellem 2 linjenumre i fil
\newcommand{\kodel}[5]{
            \lstinputlisting[style=sourceC, language=C, firstline=#4, lastline=#5, caption={#2}, label={code:#3}]{#1}
}

%Processanalyse
\renewcommand{\subsubsection}[1]{{\bf #1\\}}

%Note fix
\newcommand{\sfix}[1]{\fix{#1 /Søren}}
\newcommand{\jfix}[1]{\fix{#1 /Jimmi}}
\newcommand{\mefix}[1]{\fix{#1 /Mads}}
\newcommand{\lfix}[1]{\fix{#1 /Lars}}
\newcommand{\mifix}[1]{\fix{#1 /Mikkel}}
\newcommand{\mafix}[1]{\fix{#1 /Martin}}
\newcommand{\kfix}[1]{\fix{#1 /Kim}}
\newcommand{\lafix}[1]{\fix{LaTeX fejl: #1}}

%Dead fix
\newcommand{\sfixf}[1]{\dfix{Søren: #1}}
\newcommand{\jfixf}[1]{\dfix{Jimmi: #1}}
\newcommand{\mefixf}[1]{\dfix{Mads: #1}}
\newcommand{\lfixf}[1]{\dfix{Lars: #1}}
\newcommand{\mifixf}[1]{\dfix{Mikkel: #1}}
\newcommand{\mafixf}[1]{\dfix{Martin: #1}}
\newcommand{\kfixf}[1]{\dfix{Kim: #1}}

% Failover i tilfælde af fixme breaker...
%\renewcommand{\fixme}[1]{} % Til afleveringer
%\renewcommand{\fix}[1]{} % Til afleveringer
%\renewcommand{\dfix}[1]{} % Til afleveringer
%\renewcommand{\listoffixmes}{}                    % Kommandoer