%Skrevet af Jimmi. 20-12-2013 07:06
\subsection{Beskrivelse}
Gruppens og vejledernes samarbejdsforventninger blev indledningsvis opstillet i form af en vejlederkontrakt. Her opstillede gruppen de overordnede forventninger og dertilhørende kriterier til vejlederne, som havde til formål, at sætte nogle klare retningslinjer for, hvordan den interne kommunikation skulle foregå mest hensigtsmæssigt (Se bilag \ref{bil:kontrakt}). Ydermere blev forventningerne til gruppen, påført kontrakten. Disse krav skulle på lige fod med forventninger til vejlederne, klarlægge vejledernes forventninger til gruppen. Hver uge blev der tidslagt ét vejledermøde; hvor vejlederne ønskede en dagsorden tilsendt dagen forinden. Før vejledermøderne blev afholdt, var dagsordenen altid påskrevet tavlen, hvilket - i takt med - at gruppen nærmede sig afleveringsdeadlinen, ikke blev gjort. \\

\subsection{Analyse}
Vejlederkontrakten blev udarbejdet med stor omhyggelighed, som foruden at gavne samarbejdsvilkårene, ligeledes bidragede konstruktivt til gruppens interne forståelse af vejledernes rolle. Det var her elementært, at vejlederkontrakten - allerede inden samarbejdet påbegyndte - påtvang gruppen til at foretage nogle kritiske valg, om hvad formålet med vejlederne var. Gruppen anvendte vejlederne til at hjælpe med det rent kontekstuelle, materialesøgning, interne gøremål, opstille realistiske deadlines, fremvise nye værktøjer og sidst men ikke mindst, at skabe den røde tråd - fra rapportens begyndelse til slutning. Vejlederne tilbød en uge forinden deadline, at gruppen kunne tilsende alt materiale, som var en stor hjælp, til at få plads på de sidste ting. \\ 

Afslutningsvis blev der ikke tilsendt rapportmæssige materialer og dagsorden, hvilet gjorde at vejledermøderne mistede deres overordnede funktionalitet; at genskabe overblikket, og få gruppen på rette kurs. Dette skyldes i høj grad, at gruppen led under et massivt tidspres. Ydermere skyldtes det, at problemløsningens repræsentative afsnit ikke var færdigbearbejdet, hvilket gruppen havde fået den forståelse af, at det tilsendte materiale skulle være gennemarbejdet. \\

\subsection{Fremadrettet}
Fremadrettet skal gruppen være bedre til at bruge vejlederen, så snart der opstår spørgsmål, som gruppen ikke selv kan svare på. Gruppen skal ligeledes være meget bedre til at få tilsendt dagsordenen og dertilhørende materiale indenfor aftalt deadline. Det var rigtig godt at gruppen, allerede indledningsvis, brugte den fornødne tid på at gennemtænke hvordan vejlederne skulle bruges mest konstruktivt, hvilket gruppen også vil gøre fremadrettet.
