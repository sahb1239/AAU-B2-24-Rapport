%Skrevet af Jimmi. 19-12-2013 13:08
\subsection{Beskrivelse} %Mads
\subsubsection{Planlægningsværktøjer}
Gruppens planlægningsværktøjer blev debatteret i projektets startfase. Dette var væsentligt, for at gruppen hurtigt kunne få skitseret alle arbejdsopgaver og deadlines. Gruppen valgte at udarbejde en tidsplan i form af et GANTT-skema (Se bilag \ref{GANTT}). GANTT-skemaet blev fremstillet i et Google Spreadsheet, hvor det løbende blev opdateret i takt med at arbejdsopgaverne blev nået. Ydermere valgte gruppen at udskrive små labels, som skulle hænge fysisk på en opslagstavle i grupperummet. Dette skulle, som supplement til spreadsheetet, give en fysisk visualisering af de overordnede arbejdsopgaver. Her blev der opstillet 3 kolonner på opslagstavlen, som hver især repræsenterede opgavens status. Henholdsvis; ikke-påbegyndte, igangværende samt færdige arbejdsopgaver. Dette kaldes et scrumboard, hvilket havde til formål, at så snart man var fysisk tilstede i grupperummet, kunne gruppen ikke undgå at se på boardet – som skulle lede til en fuldkommen ajourført oversigt. \\

\subsubsection{Gruppestyreformand}
Gruppen valgte indledningsvis at udnævne en gruppestyreformand, som skulle fungere som ordfører under gruppe- og vejledermøder. Ligeledes havde styreformanden til opgave, at gavne den samlede arbejdsindsats. Under diskussioner og ved uoverensstemmelser på tværs af gruppens medlemmer, skulle styreformanden kunne bryde ind, og tage det endelig ord eller beslutning. Styreformanden skulle også sørge for at de repræsentative deadlines blev nået til tiden, samt at ajourføre GANTT-skemaet. Sidst men ikke mindst, skulle styreformanden stå for alt mailkorrespondance mellem gruppen og vejledere, for at sikre sig unødig forvirring, om hvorvidt materiale var blevet afsendt eller ej. \\

\subsubsection{Gruppekontrakt}
Arbejdsindsatsen blev inkluderet i gruppekontrakten, for at sikre sig, at aftalerne blev overholdt. Det blev erfaret under udførelsen af P0-projektet, at flere af gruppemedlemmerne havde svært ved at overholde de opstillede mødetider – hvorved gruppen ønskede indledningsvis, at forebygge denne problemstilling. \\

\subsubsection{Gruppemøder}
Hver uge blev der opstillet 2 gruppemøder. Ét i ugens begyndelse, og ét i ugens slutning. Disse gruppemøder havde til formål, at kunne uddelegere en række arbejdsopgaver i ugens begyndelse. Disse opgaver skulle enten påbegyndes, eller være færdige i løbet af ugen. Ved ugens afslutning, blev der ligeledes planlagt et gruppemøde, som skulle opsamle og ajourføre disse arbejdsopgavers status. Det var her gruppestyreformandens opgave at der blev holdt en ordentlig tone under gruppemødet, for at sikre sig at gruppemødet forbliv konstruktivt. Ydermere var gruppemøderne formål at snakke interne problemer og gøremål igennem (herunder eventuelle brud på gruppekontrakten). \\

\subsubsection{Intern kommunikation}
Under projektforløbet blev der planlagt i detaljer en uge frem, hvor arbejdsgaverne, dag for dag, blev uddelegeret til gruppens repræsentative medlemmer. Flere af gruppemedlemmerne fik sat mere for hånden, når de arbejdede hjemmefra. For at muliggøre dette, blev der opsat en række kommunikations-medier, som gruppen kunne anvende under hjemmearbejde. Der blev opsat en Facebook-gruppe, hvor alle spørgsmål, sygemeldinger m.v. blev kommunikeret til gruppens resterende medlemmer. Ydermere blev der opsat en Google Hangout gruppe, så gruppen kunne foretage internetopkald til hinanden.

\subsection{Analyse}
\subsubsection{Planlægningsværktøjer}
I takt med at gruppen nærmede sig afleveringsdeadlinen, skulle de sidste uger planlægges med stor omhyggelighed. Her blev GANTT-skemaet et irritationsmoment, idet at Google Spreadsheet er langsomt at arbejde i, indeholder flere fejl, og det gjorde hele projektplanlægningen meget uoverskuelig. I løbet af projektforløbet, havde projektet opmagasineret sig til en størrelse, som gjorde at der både skulle scrolles vertikalt såvel som horisontalt (på computerskærmen). GANTT-skemaet var blevet opbygget med gruppemedlemmernes initialer, og farver som skulle vise opgavens status. Idet at dagene var opsat vertikalt, var der ikke den fornødne plads i Spreadsheetets celler, til at initialerne var læsebare. Ligeledes skabte farverne – status-indikatorerne, blot yderligere forvirring, som til sidst ledte til, at GANTT-skemaet ikke blev anvendt af gruppen længere. \\

For at genskabe overblikket, blev der taget et nyt værktøj i brug: Trello (Se bilag \ref{Trello}). Dette planlægningsværktøj gjorde det muligt for alle gruppemedlemmerne, at påføre sig selv til ledige arbejdsopgaver. Ydermere havde værktøjet en række brugbare funktionaliteter, såsom at sætte deadlines, påføre navne, og lave uddybende beskrivelser og kommentarer under hver arbejdsopgave.
Gruppen valgte at sortere arbejdsopgaverne i nogle overordnede kategorier. Herunder blev der opstillet én kolonne, der skulle indeholde alle arbejdsopgaver til rapporten. Ligeledes blev dette gjort for programmeringsopgaverne. Trello lettede planlægningsoverblikket markant, som gjorde at gruppen fik genskabt overblikket og strukturen, hvilket havde en afdæmpende effekt på alle gruppemedlemmerne. Her fandt gruppen virkelig ud af, hvor stor betydning planlægning havde for humøret, og den samlede arbejdsindsats. \\ 

\subsubsection{Gruppestyreformand}
Under projektets start, blev samtlige deadlines - imellem gruppen og vejledere - overholdt. I takt med at gruppen begyndte på problemløsningen, skred disse deadlines mere og mere. Problemløsningen var ikke blevet gennemset af vejlederne i 2 uger, grundet at deadlines ikke var blevet overholdt. Hvilket betød, at gruppen kunne have foretaget et sidespring, som kunne have haft vitale konsekvenser for det færdige produkt. \\

\subsubsection{Gruppekontrakt}
Gruppekontrakten havde ikke den ønskede effekt på arbejdsindsatsen. De grundlæggende regler, som samtlige medlemmere havde underskrevet, blev ikke overholdt – især mødetiderne. Her udviste flere af medlemmerne at de var utilfredse med mødetiderne, og kom – på trods af det aftalte tidspunkt (8.15) – først ved 9 tiden. Dette gik ud over arbejdsindsatsen, især de dage, hvor gruppemødet var blevet lagt fra morgenstunden af. Her var det i bund og grund umuligt at holde et velfungerende gruppemøde, idet at flere af medlemmerne ikke var tilstede. \\

\subsubsection{Gruppemøder}
Gruppemøderne blev ofte forskudt et par dage, og i travle kursusperioder, blev datoerne for gruppemøderne planlagt ud fra travlhed og behov, fremfor den 2-gangs-ordning som gruppen havde opstillet. Dette betød, at deadlines ofte skred – eller ikke blev ajourført. Gentagende gange blev mødetiderne og andre problemstillinger taget op på gruppemøderne, som ikke bevirkede til nogle ændringer. Ofte var udbyttet af gruppemøderne store, og der blev diskuteret og forløst rigtige mange problemstillinger; både rapportmæssige problemstillinger, afgrænsninger såvel som interne problematikker. \\

\subsubsection{Intern kommunikation}
Facebook blev anvendt i høj grad til at kommunikere med resten af gruppen. Dette værktøj virkede efter hensigten. Facebook var godt, fordi det kunne tilgås uafhængig af fysisk placering samt enhed. Dette betød, at gruppemedlemmerne kunne deltage i diskussionerne, på trods af, at den pågældende person holdte en pause i stuen, eller allerede havde lagt sig i seng. Google Hangout havde ligeledes den ønskede effekt, idet at det var muligt at holde morgen- og aftenmøder, selvom medlemmerne arbejdede hjemmefra.

\subsection{Fremadrettet} 
\subsubsection{Planlægningsværktøjer}
Fremadrettet vil Trello helt sikkert være det foretrukne planlægningsværktøj. Erfaringerne for værktøjet var rigtige gode, og var meget hurtigere at opsætte, redigere og slette end hvad det var tilfældet med GANTT-skemaet. I stedet for at samle alle arbejdsopgaverne i ét ”Board”\footnote{Trellos definition på én opslagstavle}, vil det fremadrettet være bedre, at opdele rapportdelene i ét board, og programmeringsopgaverne i et andet. Dette vil strukturer – og særskilte hovedkategorierne yderligere, og hjælpe til at ikke få et overflod af ”Cards”\footnote{Trellos definition på en kolonne} på ét board. Ligeledes kunne det være at favorisere hvis problemløsningen og problemløsningen ligeledes var opdelt i hver deres board. Et par over-kategoriseringer vil hjælpe meget på det samlede overblik, hvorved det især ville være skønt hvis boardet havde så få cards, at den vertikale scrollbar ikke eksisterede. \\ 

\subsubsection{Gruppestyreformand} 
Gruppestyreformanden-posten skal gå på en ugentlig turnus. Hermed ville alle prøve at have ansvar, og sørge for at gruppekontrakten bliver overholdt. Dette kunne være en hjælp til enkelte personer, som hermed får ansvar – for ikke blot at overholde reglerne selv – men også at andre medlemmere overholder dem. Alle medlemmere skal ydermere deltage aktivt i at overholde vejledernes deadlines, og hjælpe styreformanden til samme. \\

\subsubsection{Gruppekontrakt}
Gruppekontrakten havde ikke den ønskede effekt, til trods for, at samtlige punkter i gruppekontrakten var diskuteret på forhånd, samt underskrevet. For ikke skabe en dårlig stemning iblandt gruppemedlemmerne, er det vigtigt at det bliver taget på et forholdsvis tidligt stadie. Det viste sig dog, at uanset hvordan og hvornår den brudte gruppekontrakten blev diskuteret, havde det ikke den ønskede effekt. Fremadrettet kan man ikke gøre andet, end at håbe at alle er indforstået med reglerne der er påført i kontrakten. \\

\subsubsection{Gruppemøder}
Til gruppemøderne er det vigtigt at alle deltager. Det er her problemstillinger kan tages i forløbet, inden de udvikler sig til irritationsmomenter, og skaber en dårlig stemning. Det vil derfor være vigtigt at opstille møde-tidspunkterne realistisk, og tage hensyn til gruppemedlemmere, som gerne vil have møderne på et senere tidspunkt. Fremadrettet skal møderne overholdes uanset hvor mange kursustimer gruppen har. Et manglende gruppemøde kan nemlig ledsage til forvirringer, svækket arbejdsmorale samt manglende opfølgning på deadlines og arbejdsopgaver. \\

\subsubsection{Intern kommunikation}
Idet at den interne kommunikation fungere efter hensigten, vil det være naturligt at benytte de samme værktøjer. Gruppen vil dog altid forsøge på at optimere den interne kommunikation yderligere, hvis der løbende forefindes nyere og mere tidsbesparende værktøjer. \\
