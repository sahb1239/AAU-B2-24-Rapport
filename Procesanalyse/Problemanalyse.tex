\subsection{Beskrivelse} %Martin
Da det valgte overemne, automatisering af hjem, var et med utallige potentielle problemstillinger, blev gruppen nødt til at afgrænse kraftigt i det fra første dag, for at nå ned til et overkommeligt arbejdsemne. Dette blev gjort ved hjælp adskillige diskussioner i gruppen, samt en brainstorm. Se figur \ref{fig:brainstorm}.

På trods af indsatsen tog det gruppen lang tid at nå ned til en endelig problemstilling at arbejde med. Dette medførte at dele af dataindsamlingen bærer præg af, at fokus ikke endnu var på den i enden valgte problemstilling.\\

%Mads
\subsubsection{Spørgeskema}
Dataindsamlingen blev konstrueret i "Survey-Xact" for at kunne behandle dataen lettere, men ligeledes at have muligheden for at sende dette rundt til forskellige fora, uden at skulle manuelt indtaste data.
Gruppen valgte at distribuerer spørgeskemaet via forskellige forums, sociale medier og til forskellige mennesker på gaden. Spørgeskemaet blev udformet relativt tidligt i dataindsamling, dette gjorde at spørgsmålene ikke var velovervejet og der senere hen i forløbet opstod spørgsmål som gruppen ikke havde noget data på. Gruppen distribuerede spørgsmålet via forums hvilket foregik relativt let, ved at 2 gruppemedlemmer undersøgte, hvilke forums som allerede kunne være forbrugere af "Smart-home" teknologien, ligeledes blev der udvalgt almene forums for at få varieret data.\\

Dataindsamlingen foregik ligeledes via interview på gaden. Dette foregik ved at 2 gruppemedlemmer stod i gågaden med 2 tablets, og spurgte forbigående folk om de ville bruge 3-5 min på at svare på dataindsamlingen.\\

Hvis folk dertil svarede at de gerne vil deltage, vil gruppen udspørge den pågældende person med de prædefinerede spørgsmål som var på tabletten.\\

Efter dataindsamlingen, valgte gruppen at bruge "Survey-Xact" til at lave grafer og statistik. Heriblandt var der flere kommentarer som skulle gennemlæses, og derved fandt gruppen en retning som kunne præge den oprindelige problemstilling. Denne kommentar omhandlede stemmestyring, som gruppen fandt meget interessant, da det ikke havde været omdiskuteret under brainstormen. Kommentargiveren havde et "smart-home", og var derfor en person som kunne sige hvad forbedringer der kunne udføres.\\ \mefix{Bilag: spørgeskema, med den der har svaret det med stemmestyring som kommentar}

Survey-Xact blev ligeledes brug til at danne overskuelige bilag og figurer i rapporten. Dette gjorde gruppen for at danne et illustrativt syn på dataindsamlingen i problemanalysen.\\


\subsubsection{Interview med Zensehome} % Søren
Gruppen valgt at lavet et interview med virksomheden Zensehome, dette blev gjort for at kunne bygge videre på et lignende system og se om systemet havde nogle specifikke mangler. Zensehome lå desuden tæt på Aalborg Universitet og var derfor oplagt for gruppen at besøge. \\

Gruppen var generelt meget imponeret over systemet, der var blevet fremstillet af Zensehome. Systemet kunne håndtere automatisk slukning af standbyapparater, hvilket i begyndelsen, var gruppens hovedformål for problemanalysen. \\

\subsection{Analyse}
\subsubsection{Spørgeskema}
Det lykkedes at samle 272 besvarelser, samt 40 ufuldendte besvarelser til dataindsamlingen, hvilket er meget godt set udfra det tidsrum, som gruppen brugte på publisering. \\

Interviewet på gaden forløb dog ikke særligt godt, folk troede at vi var sælgere og mange undgik os dermed. Derudover var tidsrummet ikke særlig optimalt, idet at der var begrænset med mennesker på gaden i det pågældende tidsrum (09:00 - 12:00). \\

Respondenterne af spørgeskemaet har generelt svaret seriøst på spørgsmålene. Der er dog enkelte useriøse svar iblandt. Ingen af de useriøse svar frasorteret, da det ville have været en kæmpe opgave at skulle 272 respondenters bevarelser igennem. \\

Survey-Xact var et udemærket værktøj til analysering af spørgeskemaets resultat. Det havde dog en ulempe i forhold til det designmæssige, hvor systemet ikke var særlig brugervenligt. For at designe et spørgeskema, der tilpasser sig efter respondenten svar, var det lettes både at benytte en simpel designer kombineret med en avanceret designer - hvilket besværliggjorde designprocessen. Udover dette var det også vanskeligt, at få fremstillet et ordentligt layout, idet at spørgeskemaet også skulle fungerer på tablets. SurveyXact er ud fra, hvad gruppen oplevede et rigtig kraftfuldt værktøj. Det kræver dog, at gruppen skulle bruge den fornødne tid på opsætning. SurveyXact er desuden pr. 19 December, kommet med udvidelsesmulighed, som gør det muligt at vælge prædefineret skabeloner til sit spørgeskema.\\

Afslutningssiden blev valgt, som en simpel side med noget tekst. Her takkede gruppen, respondenten for at have besvaret spørgeskemaet. Efter 10 sekunder blev respondenten sendt tilbage til spørgeskemaet. Denne funktion blevet lavet, idet at tablet-brugerne skulle have mulighed for at udfylde flere besvarelser på samme enhed. \\

\subsubsection{Interview med Zensehome}
Gruppens interview med Zensehome blev lavet med en sælger fra virksomheden, derfor kunne der være visse tekniske aspekter der ikke blev besvaret fyldestgørende. Derfor blev mange af oplysningerne, givet ved demonstrationen dobbelttjekket. \\

Spørgsmålene var ikke blevet opbygget efter nogen model, men ud fra individuelle erfaringer. Disse erfaringer viste sig dog, at harmornere meget godt med modellerne, som gruppen senere besluttede sig for, at inddrage i rapporten, som dokumentation for spørgeskemaet. \\

\subsection{Fremadrettet}
\subsubsection{Spørgeskema}
Et spørgeskema er en god måde at undersøge tendenserne blandt forskellige grupperinger i samfundet. Et spørgeskema kan derfor være meget relevant i et software projekt, til at undersøge generelle tendenser.\\

Det kan anbefales at sørge for at tænke mere over udførelsen af spørgeskemaet, således at spørgsmålene passer i højere grad med det problem der ønsker at løses. Derfor skal spørgeskemaet være velovervejet.\\

Derudover anbefales det at have mulighed fra at se fra hvilke kilder en given respondent kommer fra. Dette for at kunne vide hvor de fleste af svarene kommer og dermed kunne konkludere hvilke type mennesker der svarer på spørgeskemaet.\\

Spørgeskemaer på gaden er generelt en god ide, det skal dog foregå på tidspunkter hvor der er mange. Derudover skal det fremgå at vi er fra AAU fordi folk så vil vide at vi ikke er sælgere. Dette kunne gøres vha. specielle gule veste fra AAU.\\

Survey-xact var generelt et meget godt værktøj der også kan anbefales, derudover er systemet også gratis for AAU studerende. Analyseværktøjet i Survey-xact er også meget kraftfuldt, dog er selve oprettelsen af spørgeskemaet ikke særligt godt lavet, men alt i alt fås de fleste muligheder (af hvad vi har undersøgt) hos Survey-xact.\\

\subsubsection{Interview med Zensehome}
Det kan fremadrettet anbefales at lavet et interview af en virksomhed efter at et eventuelt spørgeskema er blevet fremstillet. Derudover var det en generelt god ide at høre hvad virksomhederne står for, og dermed få større indsigt i produktet de laver.


