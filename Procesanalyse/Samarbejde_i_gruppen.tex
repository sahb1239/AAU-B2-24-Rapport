%Skrevet af Jimmi. 20-12-2013 00:12
\subsection{Beskrivelse}
\subsubsection{Opgaveuddelegering}
Arbejdsfordelingen skete ud fra det standpunkt, at alle skulle have lige del i programmeringsfasen såvel som rapportskrivning. Indledningsvis fokuserede gruppen dog entydigt på problemanalysen, som gruppen ønskede at færdiggøre først. Her blev hovedemnerne uddelegeret i små grupper, bestående af 2 gruppemedlemmer i hver. Dette havde til formål, at lette forarbejdets-processen; hvorved de repræsentative grupper gik hver til sit, for at diskutere hvad indholdets skulle være. Dette blev gjort ved at anvende en D.D. (Dobbelt Disposition). Herefter blev indholdet fremlagt for resten af gruppen, som til sidst førte til en opgaveuddelegeringen i grupperummet. \\

Der blev taget nogle individuelle tests, som skulle synliggøre gruppemedlemmernes individuelle kompetencer. Her blev der blandt andet taget Belbins Teamrolle-test, for at fastslå hvilke svage såvel som stærke sider gruppemedlemmerne hver især besad. I takt med opgaveuddelegeringen, blev gruppen enige om, at det ville være uforsvarligt at basere arbejdsfordelingen på Belkins Teamrolle-testen, da det er svært at klassificere gruppens medlemmere så firkantet. Der var også enighed om at gruppens medlemmere burde komme ud af deres komfortzone og arbejde med emner, som de ikke var fortrolige med, da det er der medlemmerne kunne lære mest. Dette var ligeledes en grund til at forkaste testresultaterne, som måske ville have ledt gruppemedlemmerne til kun at arbejde med det slags materiale de var vant til. For at bedømme hvor gruppens styrker lå, blev gruppen derfor enige om at uddelegere arbejdet baseret på tilgængelighed. Så snart et medlem var færdig med et afsnit, blev et nyt uddelegeret. Dette resulterede i, at de forskellige former for afsnit blev tildelt relativt tilfældigt imellem gruppemedlemmerne. \\

Programmets funktioner blev under problemløsningens uddelegeret, således at P1-projektet skulle styrke de svage. Desforuden ønskede gruppen at programmet skulle have en høj kvalitet. Gruppen løste dette ved at opsætte en par-programmeringsform, så programmeringsopgaven kunne gå på turnus. Her blev programmets samlede funktioner opstillet i fællesskab, hvorefter gruppens bedste programmør blev sat på som supervisor. Det var supervisorens opgave at kravene blev realiseret på en hensigtsmæssig måde. Ydermere skulle de resterende gruppemedlemmerer opererer under supervisorens fremgangsmetoder. Da afleveringsdeadlinen nærmede sig, blev programmeringens-funktionaliteterne dog hovedsageligt udført af gruppens bedste programmører; dette var for at nå de opstillede krav til programmet. \\

\subsubsection{Gruppearbejdet i grupperummet} 
Det var vigtigt for arbejdsindsatsen, at der blev holdt en ordentlig tone på tværs af gruppens medlemmere. Dette var især vigtigt når gruppemedlemmerne var fysisk tilstede i grupperummet, for at undgå en faldende arbejdsindsats og morale. Gruppen led ikke under de store uoverensstemmelsesproblemer, som gjorde at gruppen deltog aktivt i rapportens udarbejdelse, i de timer, hvor gruppen var tilstede. Grupperummet blev anvendt til at afholde møder – både vejledermøder såvel som gruppemøder. Her blev grupperummets kridt-tavler brugt til at påføre dagsordner. Ydermere blev tavlerne anvendt under diskussioner, som lettede forståelsen igennem visualiseringer. Under udarbejdning af D.D’en blev tavlerne ligeledes brugt, til at påføre hovedemnernes indhold. C-Programmeringens struktur, som skulle foregå i en tilstandsbaseret opbygning, blev også påført tavlen, for at diskutere hvordan problemstillingerne kunne løses mest hensigtsmæssigt. \\

\subsubsection{Hjemmearbejde}
Der blev diskuteret på tværs af gruppen, under hvilke forhold, gruppens medlemmere skrev bedst. Der var flere af gruppemedlemmerne, som gerne ville have ro under skrivning, hvilket der blev taget hensyn til, ved at gøre det muligt at arbejde hjemmefra. Et af kravene var dog, at alt arbejde skulle ligges tilgængeligt på Cloud-tjenester, såsom gruppens Google Drive konto, eller Sharelatex – som gruppen havde tilkøbt sig adgang til. Under rapportudarbejdelsen led Sharelatex dog under massive fejl, såsom compiling-errors, store forsinkelser på compiling samt periodiske fejl hvor det ikke muligt for gruppens medlemmere at logge ind. 

\subsection{Analyse}
\subsubsection{Opgaveuddelegering} 
Gruppens 2-mands-grupperinger fungerede efter hensigten. Her blev der virkelig sat noget for hånden, og på blot 1½ time havde gruppen konstrueret en ekspose over samtlige arbejdsopgaver. Gruppen skulle muligvis have brugt Belbins Teamrolle-testen i højere grad, for derved at kunne lave de korrekte rollefordelinger allerede indledningsvis. Af hensyn til at opnå forøget kompetence, viden og erfaring var gruppen bevist om sine valg omkring uddelegering af programmeringsopgaverne. Par-programmeringsformen var et godt værktøj til at balancerer disse ønsker – høj kvalitet og forøget kompetence. \\

\subsubsection{Gruppearbejdet i grupperummet} 
Grupperummet blev anvendt som det skulle. De supplerende tavler, som kunne bruges henholdsvis som opslagstavle og almindelige kridttravle blev anvendt efter hensigten. Rent oprydningsmæssigt kunne gruppen dog have været bedre til. Et uorganiseret grupperum ledsagede nemlig til rodede papirer, flaskepant samt et utal af brugte kaffekopper. \\

\subsubsection{Hjemmearbejde} 
De fleste af gruppens medlemmere fik brugt hjemmearbejdet som en konstruktiv arbejdsform. Det viste sig dog, at flere have svært ved at holde sig selv i nakken, hvilket naturligvis gik udover den enkelte medlems arbejdsindsats. Hjemmearbejdet viste sig i høj grad, at den hjemmearbejdende skulle have selvjustits. Flere gange blev materialet ikke lagt op på Cloud-tjenesterne, selvom dette var aftalen, hvilket betød, at de resterende medlemmere havde svært ved at sammenskrive afsnittet med deres egne. Dette gik beklageligvis udover den samlede kvalitet, og ofte betød det, at flere af områder skulle omskrives – for at bevare den røde tråd, og undgå irritable gentagelser. Sharelatex blev brugt, da det havde de ønskede funktionaliteter; kan anvendes på tværs af gruppens medlemmere, og i en real-time-editor, hvorved gruppen kunne blive fri for at anvende yderligere programmer, såsom eksempelvis versionsstyring. 

\subsection{Fremadrettet}
\subsubsection{Opgaveuddelegering} 
Idet at opgaveuddelegeringen foregik hensigtsmæssigt, var der ikke det store behov for yderligere udredelighed. Fremadrettet skulle gruppen måske være bedre til at uddelegere programmeringsopgaverne, så det ikke er de bedste programmører der sidder de sidste par uger og færdiggøre programmet. Dette er dog i højere grad et spørgsmål om at have et ordentlig planlægningsværktøj fra begyndelsen. \\

\subsubsection{Gruppearbejdet i grupperummet} 
Fremadrettet vil gruppen være bedre til at lave regulære oprydninger. Dette kan gøres ved at ligge en fast dag for hvornår oprydningen skal ske, og ligge dette ind som en fast dato i planlægningsværktøjet. \\

\subsubsection{Hjemmearbejde} 
Gruppen skal være bedre til at snakke om de enkelte medlemmers arbejdsindsats i hjemmet. Her skal der ligges nogle faste rammer for hvornår det er mest hensigtsmæssigt at arbejde hjemme, og hvornår tingene skal foregå i grupperummet. Dette er meget afhængigt af personlige præferencer, som der ligeledes – fremadrettet – bør tages hensyn til. På trods af de store og irritable fejl, ønsker gruppen at anvende Sharelatex igen til næste projektudarbejdelse. Dette skyldes, at den samlede tidsbesparelse på at have hele rapporten samlet ét sted, vil være større end hvis gruppen skulle påbegynde versionsstyring. Gruppen skal dog være bedre til at anvende Sharelatex som primære arbejdsværktøj, for derved at blive fri for at noget materiale ikke er tilgængelig for de andre medlemmere.





% Martin: Alt det udkommenterede minder for meget om det der er skrevet i projektplanlægningen.
%I starten af samarbejdet skrev vi en gruppekontrakt, som var så ekstensiv som muligt, for at formindske chancen for uforudsete problematikker ved gruppearbejdet. \mafix{Ref til gruppe-kontrakt i bilag.}Under denne kom vi blandt andet ind på:\\
%\begin{itemize}
%    \item Ambitionsniveau
%        \begin{itemize}
%            \item Det forventedes at gruppens medlemmer prioriterede studiet og projektet højt, indbefattende at alle kom til relevante forelæsninger og gjorde en indsats for at lære de fornødne egenskaber nødvendige for at bistå gruppen, såsom c-programmering og latex-teori.
%            \item Vi blev enige om at arbejde hen imod at skrive en rapport af middel-høj kvalitet, resulterende, forhåbentligt, i en karakter i den høje ende af skalaen.
%        \end{itemize}
%    \item Gruppe-møder
%    \begin{itemize}
%        \item Vi valgte som udgangspunkt at holde to gruppemøder om ugen, med formål at samle op på nyligt lavet arbejde, uddelegere nyt arbejde og kæde de forskellige afsnit sammen, således at projektets røde tråd blev så god som muligt igennem forløbet. Hvis problemer i gruppen var opstået blev disse ligeledes diskuteret. Disse møder blev afviklet således:
%        \begin{itemize}
%            \item En bred dagsorden blev skrevet af gruppelederen, eller erstatning, før mødet påbegyndtes.
%            \item Dagsordenen blev gennemgået slavisk. Hvis diskussionen gik i stå på et punkt eller folk blev useriøse var det gruppelederens job, at få diskussionen på rette vej igen eller dømme emnet uddebateret og fortsætte på næste punkt.
%            \item For hvert møde blev et referat skrevet løbende, af en sekretær som blev valgt forud for mødet.
%        \end{itemize}
%    \end{itemize}
%    \item Konfliktløsning
%    \begin{itemize}
%        \item For at imødegå eventuelle konflikter imellem gruppens medlemmer blev der ikke bestemt noget specifikt, da konflikter altid har specifikke omstændigheder, som er svære at forudse. Det blev besluttet, at hvis en konflikt måtte opstå tages denne på et gruppemøde, hvor gruppelederen har det sidste ord, såfremt han er upartisk. 
%    \end{itemize}
%\end{itemize}
%
%Arbejdsopgave fordelingen skete ud fra det standpunkt, at alle skulle have lige del i programmerings- og rapportskrivning, hvilket resulterede i den arbejdsfordeling beskrevet i foregående afsnit. Udover dette var der ikke specielt fokus på en arbejdsfordeling baseret på gruppemedlemernes individuelle evner. Vi blev påbudt at tage et antal tests, såsom Belbins teamrolle test\mafix{Eventuelt kilde her, men så skal vi have et litteratur afsnit}, for at fastslå hvilken rolle gruppemedlemerne hver især burde udfylde. Gruppen blev dog enige om, at det ville være uforsvarligt at basere arbejdsfordelingen på en sådan test, da det er svært at klasificere personer så firkantet. Der var også enighed om at gruppens medlemmer burde komme ud af deres komfortzoner og arbejde med emner, som de ikke var fortrolige med, da det er der man lærer mest. Dette var ligeledes en grund til at forkaste testresultaterne, som måske ville have ledt gruppemedlemmerne til kun at arbejde med det slags materiale de var vante til. For at bedømme hvor gruppens styrker lå, blev gruppen derfor enige om at uddelegere arbejdet baseret på tilgængelighed. Så snart et medlem var færdig med et afsnit, blev et nyt uddelegeret. Dette resulterede i, at de forskellige former for afsnit blev tildelt relativt tilfældigt mellem gruppemedlemmerne i starten. Senere, i slutningen af forløbet, var det blevet tydeligt hvor gruppemedlemernes individuelle styrker lå, så der blev arbejde uddelegeret efter færdighed. \\




%Der har ikke været behov for at arbejde intensivt på gruppens samarbejde, da gruppens medlemmer hele tiden har haft det godt sammen socialt, og har haft gensidig respekt, resulterende i at folks synspunkter er blevet hørt, hvis de har talt op. Der har dog været problemer, hvis respekten mellem gruppemedlemmerne slap op. For eksempel stoppede et gruppemedlem et par uger før afleverings-deadline, og før det brød han aftaler. På grund af dette slap respekten op og spøg på hans bekostning, også når han var tilstede, blev lidt for normalt. Det hænger sammen med at vi er relativt ambitiøse, samt har individualistiske personligheder, hvilket vil sige, at vi ikke er glade for at bære andre igennem et forløb. \\

%Der har ikke været nogle samlede arrangementer, hvor gruppen har sat sig ned og aftalt at samles til aftensmad eller lignende. Der har været en del små samlinger efter gruppearbejde, hvor nogle af gruppemedlemmerne har sat sig på en bar eller lignende, eller hængt ud sammen i weekender.

%{\bf Analyse}
%Det er et problem for gruppearbejdet, at vi har det med at lave spøg på bekostning af dem, som vi føler ikke trækker sin vægt ordentligt. Der er flere problemer ved det:
%\begin{itemize}
%    \item Gruppearbejdet lider. Man burde altid prøve at få det bedste ud af en situation. I dette tilfælde ville dette være, at arbejde videre som om ingenting er galt, når personen er til stede. Se sagen fra en anden side. Arbejd ud fra den forudsætning, at han ikke møder op, eller ikke får lavet noget. Hvis han intet får lavet, har man taget det med i sine beregninger. Hvis han får lavet noget, er det en uventet, kærkommen hjælp.
%    \item Personen får det dårligt. Det siger sig selv, at det efterlader aftryk hvis nogen bliver bagtalt eller behandlet dårligt. Udover at dette selvfølgelig påvirker hans arbejde negativt, hvilket er dårligt for gruppen, så påvirker det også resten af hans hverdag. Dette skal undgåes, da det aldrig er i orden at påvirke en person på den måde, af nogen grund. Så vigtigt er projektet ikke. Vi er alle mennesker, med vores personlige fejl og mangler, og vi bør behandle hinanden derefter.
%    \item Usikkerhed. Hvis det bliver hverdag, at personer som ikke er lige så produktive som de andre i perioder bliver bagtalt, så kan det lynhurtigt vende sig imod alle. Hvis man har problemer, sygdom eller dødsfald i familien, har man brug for support fra sine nære venner, ikke bagvask. Gruppen kan kvalificeres således, da vi sidder sammen i mange timer, næsten hver dag.
%\end{itemize} 

%Det er et problem, at der ikke har været samlede gruppe-arrangementer, da dette kan resultere i, at der til en hvis grad bliver dannet kliker i gruppen, eller at folk bliver udeladt. Dette har ikke været et problem indtil nu, men man kan lige så godt tage problemer før de opstår.

%{\bf Fremadrettet}\\
%For at bekæmpe eventuelle problemer med en person, som ikke trækker sit læs, kan der gøres følgende, eller en kombination heraf:
%\begin{itemize}
%    \item Hjælpe personen med sit stof på en konstruktiv måde, såfremt han selv ønsker at forbedre sig. Det er ikke muligt at hjælpe en person som ikke vil have hjælp.
%    \item Tale med personen omkring det. Hvis der personlige problemer indblandet, kan det måske løses ved at lade personen bruge lidt længere tid på forskellige opgaver.
%    \item splitte gruppen. Det er hårdt, men det er bedre end en stærkt dysfunktionel gruppe. Dette burde dog være sidste udvej, siden gruppen mister et medlem, men stadig skal aflevere det samme. Dette skulle kun gøres, hvis det fastslåes, at gruppen mister mere potentiel arbejdstid ved at at beholde ham, end at lade ham gå.
%\end{itemize}
%For at bekæmpe klikedannelser skal de personer, som laver ting sammen være bedre til at invitere de andre med. Samtidig skal de personer, som ikke normalt tager med på sociale ture, overveje at tage med mere. Gruppen kan ligeledes være bedre til at lave arrangementer i fritiden, så hele gruppen samlet kommer ud og lave noget socialt sammen, resulterende i at alle lærer hinanden at kende sammen.


%Det var uden tvivl vigtigt at få udarbejdet en stor gruppe-kontrakt, således at problemer kunne tages mens de opstod. For eksempel, en diskussion om mødetider, som blev bragt op af et gruppemedlem, som ønskede at møde kl. 9 i stedet for kl. 8.15. Resten af gruppen argumenterede imod, da den normale arbejdsdag i Danmark starter ved 8-tiden, samt at der stod i gruppekontrakten, som vi alle havde underskrevet, at vi som udgangspunkt møder kl. 8.15 på dage vi har aftalt at sidde i grupperummet.\\

%For at undgå, at en diskussion omkring gruppe-kontrakten opstår igen, er det nok nødvendigt at lægge mere vægt på gruppe-kontraktens endelighed. Dermed sigende, at diskussioner omkring gruppe-kontraktens punkter skal ske før alle underskriver den. Efter den er underskrevet er den fastlagt. Medmindre der fremkommer nye oplysninger, som gør en del af kontrakten forældet, følges kontrakten. Den er selvfølgelig ikke så absolut, at den ikke kan ændres hvis der er tydelige fejl eller mangler, men som udgangspunkt er kontraktens ord lov.
