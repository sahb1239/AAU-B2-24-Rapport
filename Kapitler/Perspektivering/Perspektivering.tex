%Skrevet af Jimmi.
%Sidst rettet: 07-12-2013, 17:57, Jimmi
%Kommatering- og stavefejls rettelse: 08-12-2013, 22:52, Martin
%Rettelse: 16-12-2013, 12:52, Martin
Stemmestyringsteknologien har siden sin introduktion gennemgået en eksponentielt stigende udviklings hastighed. Denne forbedring har ført til, at flere producenter nu anvender stemmestyring, som en aktiv teknologi i deres produkt. Hjemmet har ikke oplevet den samme form for standardiserede integration af stemmestyring, hvilket blot er et spørgsmål om tid. I takt med, at automatiseringsproducenterne kan tilbyde bedre og mere funktionelle løsninger, øger dette også kundernes interesse for at få systemet integreret. \\
Mulighederne for stemmestyring er uendelige, og stopper kun dér, hvor den enkelte udvikler sætter sin begrænsning. I et automatiseret hus, hvor en PC-boks allerede håndtere størstedelen af de elektroniske apparaturer, ville stemmestyring som supplement - forekomme som en naturlig integration. Lysdæmperne, komfuret, vaskemaskinen, samt andre elektroniske apparater, ville kunne styres igennem simple kommandoer - sagt med stemmen. Ydermere ville de elektroniske apparater, ligeledes kunne give brugerne en tilbagemelding igennem , når deres opgave var færdige. Dette kunne f.eks. være vaskemaskinen, der kunne give en kort besked, om at tøjet nu er vasket. \\
Det kan forventes at stemmestyringsteknologien som helhed, i takt med at teknologien bliver mere udbredt integreret, vil blive udviklet yderligere. Her kunne man forestille sig, at en hybrid stavekontrolsløsning af arten \textit {redigeringsafstand} samt \textit{regelbaserede sandsynlighedsberegning} ville blive effektueret. Dette ville gøre systemet i stand til at lærer af sine fejl, ved at tælle hyppighed af specifikke fejlmønstre i kommandoerne. Systemet vil ved usikkerhed ikke kun fungere dynamisk og rationalistisk, men også autokorrigerende - og foretage handlingerne på baggrund af tidligere fejl.
Det kunne forestilles, at producenter i fremtiden vil udvide tilgængeligheds-mulighederne, så stemmestyringen vil kunne benyttes uafhængig af brugerens geografiske placering. Her kunne man forestille sig controllere i bilen, på cyklen eller på arbejdspladsen, hvorved tidsbesparelsen og den øgede komfort ikke kun vil være imens brugerne er hjemme, men også i livets andre aspekter.\\

I forhold til hvordan gruppens løsning spiller ind i den videre udvikling af stemmestyring, må det desværre antages at det ikke er meget. På grund af tidsbegrænsningen, den økonomiske begrænsning og manglende erfaring med programmering, kan produktet ikke forventes at være i nærheden af den kvalitet, som andre udviklere af stemmestyring har opnået. Men gruppen har dog bevist, at det er muligt, på trods af førnævnte begrænsninger, at konstruere et C-program, som kan stå for stemmestyringen af et hjem, hvilket måske vil kunne hjælpe nogen. Ligeledes er der blevet samlet meget relevant information omkring stemmestyring, som måske kunne være med til at informere læsere af denne rapport omkring fordelene ved stemmestyring, og dermed potentielt skabe mere efterspørgsel. Dette skaber et mere attraktivt marked for firmaer at vove sig ind i, hvilket skaber mere konkurrence og, forhåbentligt, større udvikling.


%Skrevet af Jimmæææh.
%Sidst rettet: 07-12-2013, 17:57, Jimmi
%Stemmestyringsteknologien har siden sin introduktion, gennemgået en radikal forbedringsvis udvikling. Denne forbedring har ført til, at flere producenter nu anvender stemmestyring, som en aktiv teknologi i deres produkt. Hjemmet har ikke oplevet den samme form for standardiseret integration af stemmestyring, hvilket blot er et spørgsmål om tid. I takt med, at automatiseringsproducenterne kan tilbyde bedre og mere funktionelle løsninger, øger dette også kundernes interesse for at få systemet integreret. \\\\
%Mulighederne for stemmestyring er uendelige, og stopper kun dér, hvor den enkelte udvikler sætter sin begrænsning. Et automatiseret hus, hvor en PC-boks allerede håndtere størstedelen af de elektroniske apparaturer, ville stemmestyring som supplement – forekomme som en naturlig integration. Lysdæmperne, komfuret, vaskemaskinen samt andre elektroniske apparater, ville kunne styres igennem simple kommandoer - sagt med stemmen. Ydermere ville de elektroniske apparaturer, ligeledes kunne give brugerne en tilbagemelding, når deres opgave var færdige. Dette kunne f.eks. være vaskemaskinen, der kunne give en kort besked, om at tøjet nu er vasket. \\\\
%Stemmestyringsteknologien som helhed, vil i takt med at flere og flere, får teknologien integreret, ligeledes blive udviklet yderligere. Her kunne man forestille sig, at en hybrid stavekontrols-løsning (\textit {Redigeringsafstand} samt \textit{regelbaserede sandsynlighedsberegning}) ville blive effektueret. Dette ville gøre systemet i stand til, at lære af sine fejl, ved at tælle hyppighed og specifikke fejlmønstre i kommandoerne. Systemet vil ved usikkerhed, ikke kun fungere dynamisk og rationalistisk, men i stedet fungere autokorrigerende - og foretage handlingerne på baggrund af tidligere fejl. \\\\
%Det kunne forestilles, at i fremtiden vil producenterne udvide tilgængelighedsmulighederne, så stemmestyringen vil kunne benyttes uafhængig af brugerens geografiske placering. Her kunne man forestille sig controllere i bilen, på cyklen eller på arbejdspladsen, hvorved tidsbesparelsen og den øget komfort vil være imens brugerne er hjemme, men også i livets andre aspekter. 


%Skrevet af Martin
%Stemmestyring af enheder og hjem er et felt med stort fremtidspotentiale. Efterhånden som det bliver mere almindeligt at have et centralt system som styrer huset kan det forventes, at flere elektroniske enheder, såsom komfur, fjernsyn, mikro-ovn, lys og sågar bil bliver designet således at husets styresystem kan kontrollere dem, således at man ikke bare kan tænde og slukke for apparater, men også styre dem mere nuanceret. For eksempel, skrue op og ned for kogeplade varme og lysintensitet, skifte kanaler på fjernsyn og tænde varmen i bilen fem minutter før man går ud af døren. Ligeledes kunne enheder måske sende informationer tilbage til systemet. Komfuret kunne for eks. melde til hjemmestyringen at nu koger vandet, hvorefter hjemmestyringen kunne sige det til beboeren. I samme stil kan man også forestille sig, at det bliver muligt at komme i kontakt med sit hjemmestyrings system gennem flere medier, såsom mobiltelefon og bil, så du kan fortælle ovnen at den skal tænde sig fra gaden eller bilen.\\

%Som teknologien bliver mere udviklet kan man også forestille sig, at stemmestyringen bliver mere intelligent. Da det er et system som skal styre et hjem efter en families smag er det essentielt, at systemet er i stand til at tilpasse sig en families handlings- og stemmemønster. For eksempel spørger vores program om den af stavekontrollen fundne korrektion er den mente hver gang der opfanges en stavefejl. For at lave et mere intelligent system kunne man tælle hvor hyppigt i en tidsperiode en person laver den samme indtalingsfejl. Hvis det sker meget hyppigt kan det konkluderes, at dette bare er den måde personen udtaler det ord på, og så skal systemet automatisk ændre ordet til det korrekte hver gang den hører det.\\

%Det er selvfølgelig ikke kun i hjem stemmestyring kommer til at råde. Alle steder hvor elektroniske enheder er, vil stemmestyring efterhånden også være. Især når kunstig intelligens bliver udviklet mere bliver det påkrævet at stemme systemerne virker godt, så folk med arbejde hvor de ikke har hænderne fri, såsom politifolk på patrulje, brandmænd, soldater i kamp og håndværkere konstant kan få assistance af deres kunstige intelligens. Det er i hvert fald en ret sikker antagelse at stemmestyring bliver en stor del af de næste generationer. Så det kan betale sig at fokusere på det nu.
