%Skrevet af Martin og Jimmi.
%Sidst rettet: 17-11-2013, 12:55, Jimmi
Igennem tiderne har huset været et vigtigt element for menneskets overlevelse. Siden menneskets spæde start – i Afrika for ca. 120.000 år siden\cite{HumanOrigen}, var det ikke livsnødvendigt for mennesket at bygge et ly. Det havde selvfølgelig nogle bemærkelsesværdige fordele; beskyttelsen mod farlige rovdyr og den kolde nattekulde (det var lige op til den seneste istid)\cite{RecentIceages}. Det var ikke livsnødvendigt, som det blev senere, da grupperinger af vores forfædre udvandrede fra Afrika\cite{HumanOrigen}. De mere nordliggende steder på kloden var hårdere ramt af nattekulden, og det var her elementært for menneskets overlevelse, at kunne holde kroppen varm i løbet af natten. Dette var den primære grund til, at huset konstant blev udviklet på.\\

Meget senere, i det gamle Romerrige, blev der gravet brønde, som var tilsluttet akvædukter – dette var en vellevned som primært kun velhavende romere havde råd til. Dette havde en markant fordel, idet at man ikke længere var nødsaget til at gå udenfor, når man skulle hente vand eller på toilettet. Romernes forøgede adgang til vand i hjemmet resulterede i øget sundhed og livskvalitet samt velstanden.\cite{VandRom} \\

Omkring afslutningen af det nittende århundrede, var de dage hvor huset kun tjente ét formål - nemlig overlevelsen fra nattekulden og rovdyr - for længst ovre. Kvaliteten af et hjem blev ikke længere kun afgjort af hvor godt det var isoleret, men også hvor økonomisk besparende, komfortabelt og miljøvenligt det var. Antallet af hverdagsting man kunne udføre i sit hjem var støt stigende, op til en sådan grad, at det var unødvendigt at forlade huset, for andet end at tage på arbejde og hente mad. Dette var hovedsageligt på grund af den teknologiske udvikling, som bragte forskellige hjælpemidler ind i menneskers liv, såsom opvaskemaskinen, vaskemaskinen og ovnen. Disse hjælpemidler repræsenterede en markant tidsbesparelse og komfortforhøjelse. For eksempel, kunne tøjvaskningsprocessen, før vaskemaskinens indtog, tage en hel dags hårdt arbejde\cite{Houseworklate19th}. Elektricitetens indtog forhøjede effektiviteten yderligere og resulterende i at det efterhånden var muligt, at forlade maskinerne mens de gjorde langt størstedelen af arbejdet.\\

%Den eksplosive teknologiske udvikling i nyere tid\cite{Moore'sLaw} bragte en lang række nyskabende systemer og teknologier på banen, heriblandt stemmegenkendelsessystem, Audrey, som kom i året 1952, udviklet af Bell Laboratories. Audrey var to meter høj, ekstrem omkostningsfuld, forbrugte massive mængder strøm og var vanskelig at vedligeholde. Den have den egenskab at kunne modtage enkelte cifre fra "designerede" talere, dvs. mennesker som talte meget klart, konsistent og i en helt speciel frekvens og styrke\cite{BellLabsAudrey}. I de efterfølgende år blev teknologien videreudviklet, og der findes til dato en stribe af velkendte stemmestyringssystemer; som f.eks. Apples Siri, som ikke alene kan modtage alle ord men agere ud fra dem\cite{SiriFeatures}.\\

Hjemmet er stadig under en eksplosiv udvikling. I dag indeholder hjemmet et utal af elektroniske apparater, og i takt med at mennesket ønsker optimering i livets mange forskellige aspekter, ønskes hjemmet ligeledes optimeret. Husejerne har fået øjnene op for fordelene ved et centralt og intelligent styret system, som kan være med til at gøre dagligdagen for den enkelte husejer lettere\cite{ZensehomeKunder}. Automatisering i hjemmet er dog ikke hvermandseje, så noget tyder på; at systemerne stadig rummer vitale fejl og mangler - men hvilke?
%Og hvilke løsninger forøger tidsbesparelsen i det automatiserede hjem?

%I takt med elektricitets indtog omkring det 19. århundrede\cite{Elhistorie}, steg hjemmets teknologiniveau yderligere. Elektriciteten blev ganske vist opdaget tilbage i det 17. århundrede, men det havde ikke en relevans for det almindelige hjem, før Edison fremviste glødepæren i året 1879\cite{Cubus}. Denne opfindelse gjorde det muligt for den enkelte husejer, at hurtigt og på en enkel måde få lys i huset. Glødepærens introduktion ledsagede til en eksplosiv udvikling i elektroniske apparaturer\cite{Moore'sLaw} – hvilket revolutionerede husejernes indgangsvinkel imod et mere komfortabelt og automatiseret hjem\cite{ElektricitetHistorie}. \\