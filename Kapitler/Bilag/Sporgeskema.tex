\label{sec:sporgeskema}
% Oprettelse af specialfigur for spørgsmål
\newcommand{\figurspg}[4]{
		\begin{figure}[H] \centering \em
			\includegraphics[width=#4\textwidth]{#1}
			\caption{#2}\label{spg:#3}
		\end{figure} 
}

\section*{Læsevejledning til spørgeskema} 
Bilaget for spørgeskemaet, er herunder vedhæftet. Indledningsvis er samtlige spørgsmål opsat - herunder deres besvarelsesmuligheder. Efterfølgende er de væsentlige resultater opstillet i form af relevante diagram-typer som cirkeldiagram og søjlediagram, for at kunne give en bedre visualisering af resultaterne. \\

\section*{Det udsendte spørgeskema} 
{\bf{Introduktion til respondenten}} \\
Vi er en gruppe Softwareingeniør-studerende på Aalborg Universitet, som har brug for din hjælp til at belyse anvendelsen såvel som problematikken forbundet med automatiseringer af hjemmet.\\
Et automatiseret hjem, er en betegnelse for et intelligent el- og/eller varmeregulerende-system i boligen. Automatisering af hjemmet er blevet populært, idet at boligejerne ønsker systemernes mange fordele. Heriblandt øget komfort, økonomiske besparelser, såvel som øget fleksibilitet.\\
For at kunne besvare spørgeskemaets 16 spørgsmål, er det ikke påkrævet at du selv har ét automatiseret hjem, derfor kan du – omend du har kendskab til automatiserede løsninger til hjemmet, eller ej, stadigvæk kunne bidrage konstruktivt til projektets markedsanalyse.\\
Spørgeskemaet er konstrueret således, at det blot tager 2 minutter at besvare, og alle besvarelser vil være anonyme.\\\\
{\bf{Spørgsmål 1: Hvad er dit køn?}}
\begin{itemize}
    \item Mand
    \item Kvinde
\end{itemize}

{\bf{Spørgsmål 2: Hvad er din alder?}}
\begin{itemize}
    \item Under 15 år
    \item 15 - 17 år
    \item 18 - 21 år
    \item 22 - 28 år
    \item 29 - 35 år
    \item 36 - 49 år
    \item Over 50 år
\end{itemize}

{\bf{Spørgsmål 3: Hvad er din nuværende primære beskæftigelse?}}
\begin{itemize}
    \item Selvstændig erhvervsdrivende
    \item Lønmodtager i den private sektor
    \item Lønmodtager i den offentlige sektor
    \item Arbejdsledig
    \item Hjemmegående husmor/husfar
    \item Gymnasial uddannelse
    \item Studerende på videregående uddannelse
    \item Pensionist
    \item Andet
\end{itemize}
Disse spørgsmål er med for at hjælpe med at katalogisere vores data.\\

{\bf{Spørgsmål 4: Ejer du en bolig eller lejer du en bolig?}}
\begin{itemize}
    \item Ejer
    \item Lejer
\end{itemize}

{\bf{Spørgsmål 5: Hvilken boligtype bor du I?}}
\begin{itemize}
    \item Paracelhus / villa
    \item Lejlighed
    \item Kollegie
    \item Andet
    \item Ved ikke
\end{itemize}

{\bf{Spørgsmål 6: Hvornår er boligen fra?}}
\begin{itemize}
    \item Ældre end 1990
    \item 1990 - 2000
    \item 2001 - 2010
    \item Nyere end 2010
    \item Ved ikke
\end{itemize}

Disse spørgsmål er meget releavante pga. at vi ikke tror at folk der lejer en bolig vil have interesse i at gå ud og købe dyre komponenter til at lave huset om til et inteligent hus, derfor vil vores primære målgruppe også være folk der ejer en bolig. Ligeledes undersøges der om boligtype og alder spiller ind i interessen for et automatiseret hjem. Det må for eksempel forventes at beboere i et parcelhus/villa er mere interreserede i at automatisere deres hjem end folk i lejligheder er da de første er mere "permanente".\\

{\bf{Spørgsmål 7: Hvor mange personer bor i husstanden?}}
\begin{itemize}
    \item Bor alene
    \item 2
    \item 3 - 5
    \item Over 5
\end{itemize}

Er relevant fordi folks interrese for automatisering af hjem måske ændrer sig når der er flere brugere der skal lære det. Selvfølgelig er der så også flere der kan få gavn af det.\\

{\bf{Spørgsmål 8: I hvor mange timer om dagen, står din bolig tom? (Dvs. at ingen er hjemme)}}
\begin{itemize}
    \item 1 - 3 timer\item 4 - 6 timer\item 7 - 9 timer\item Over 10 timer\item Ved ikke
\end{itemize}

{\bf{Spørgsmål 9: Marker følgende elektroniske apparater du slukker på stikkontakten, for at undgå unødig standby-tid?}}
\begin{itemize}
    \item Spillekonsol
    \item Stereo
    \item Computer
    \item Printer
    \item Computerskærm
    \item Højtalere
    \item TV/HIFI/Video/DVD
    \item Router
\end{itemize}

Dette spørges om for at determinere om der er et problem at finde i spild af standby strøm. Hvis besvarerer angiver hvilke elektriske apparater de ikke slukker for når de tager hjemmefra og hvor længe huset er tomt hver dag, vil det være muligt at udregne hvor meget el der kan spares ved at huset selv slukker for de relevante kontakter.\\

{\bf{Spørgsmål 10: Bliver dit hjem på nogen måde automatiseret (eks. tænd og sluk af stikkontakter på bestemte tidspunkter eller mulighed for fjernstyring vha. smartphone)?}}
\begin{itemize}
    \item Ja\item Nej\item Ved ikke
\end{itemize}

Spørgsmålet her skal finde de personer der allerede ejere et automatiseret hjem og derefter vise spørgsmål 11 - 15 hvis de har det.\\

{\bf{Bemærk spørgsmål 11 - 15 bliver kun vist hvis der i spørgsmål 10 bliver svaret ja}}\\
{\bf{Spørgsmål 11: Marker hvilken/hvilke producenter der benyttes til dit automatiserede hjem?}}
\begin{itemize}
    \item Zensehome
    \item Z-wave
    \item IHC
    \item Belkin
    \item Andet
    \item Ved ikke
\end{itemize}

For at undersøge om der findes problemer at arbejde med i de forskellige producenters løsninger.\\

{\bf{Spørgsmål 12: Marker hvilke af følgende funktioner som styres automatisk?}}
\begin{itemize}
    \item Kaffemaskine
    \item Stikkontakter
    \item Varmestyring
    \item Andet
    \item Ved ikke
\end{itemize}

Spørgsmålet her skal belyse hvilke funktioner der bliver styret automatisk I besvarerens automatiserede hjem.\\

{\bf{Spørgsmål 13: Hvordan syntes du at dit automatiserede system fungerer?}}
\begin{itemize}
    \item Meget godt
    \item Godt
    \item Hverken eller
    \item Dårligt
    \item Meget dårligt
\end{itemize}

For at finde ud af hvad personer med automatiserede huse generelt mener om deres system og om hvordan det fungerer.\\

{\bf{Spørgsmål 14: Er der mangler i dit automatiserede system}}\\
Formålet er her at finde ud af om brugere af automatiserede hjem generelt syntes de mangler funktioner i deres nuværende system.\\

{\bf{Spørgsmål 15: Kunne du tænke dig at flere af dine enheder kunne snakke sammen i dit nuværende system?}}
\begin{itemize}
    \item Ja - indtast hvilke
    \item Nej
\end{itemize}

Formålet ved dette spørgsmål er at forsøget at skabe flere ideer til hvad der kunne automatiseres.\\

{\bf{Spørgsmål 16: Ville du have interesse for et inteligent system, som automatisk tilpasser dit hjems varme og elektronsiske apparater?}}
\begin{itemize}
    \item Ja
    \item Nej
    \item Ved ikke
\end{itemize}

Spørgsmålet skal finde de personer der har interesse for et automatiseret hus.\\
{\bf{Bemærk spørgsmål 17 - 19 bliver kun vist hvis der i spørgsmål 16 bliver svaret ja}}\\
{\bf{Spørgsmål 17: Hvilken type automatisering ville du personligt priotere højest, for at en automatiseret løsning ville være releavant for dig? (I høj grad, I nogen grad, Hverken eller, I mindre grad, Overhovedet ikke, Ved ikke)}}
\begin{itemize}
    \item Køkkenapparatur som eks\. automatisk brygning af kaffe
    \item Automatisk slukning af standbyappaaratur, for at spare på elregningen
    \item Varmestyring i form af automatisk regulering
    \item Sensorer i form af automatisk lys
    \item Automatiske gardiner
\end{itemize}

For at finde potentielle problemmer at arbejde med.\\
{\bf{Spørgsmål 18: I hvor høj grad vil du priotere følgende fordele ved et automatiseret hus? (I høj grad, I nogen grad, Hverken eller, I mindre grad, Overhovedet ikke, Ved ikke)}}
\begin{itemize}
    \item Økonomisk besparende
    \item Miljøvenligt
    \item Tidsbesparelse
    \item Sikkerhedsforøgende
\end{itemize}

Spørgsmålet stilles for at fastslå hvilke fordele ved et automatiseret hjem der er vigtigst for besvarere, og derved mest relevant for os at finde en problemstilling til.\\

{\bf{Spørgsmål 19: Ville du være villig til at investere i et automatiseret system, hvis det først tjener sig ind efter 10 år?}}
\begin{itemize}
    \item Ja
    \item Nej
    \item Ved ikke
\end{itemize}

En forlængelse af sidste spørgsmål.\\
{\bf{Bemærk spørgsmål 20 bliver kun vist hvis der i spørgsmål 16 bliver svaret nej}}\\
{\bf{Spørgsmål 20: Hvorfor ville et automatiseret hus ikke være interesant?}}
\begin{itemize}
    \item Overvågning
    \item Dyrt
    \item Jeg er glad for den løsning jeg har
    \item Andet
\end{itemize}

Spørgsmålet stilles for at opdage eventuelle mangler i et automatisk system som en mulig problemstilling.\\
{\bf{Spørgsmål 21: Har du gode idéer til hvad et intelligent system skulle kunne gøre?}}
Spørgsmålet her er til at finde eventuelle alternative problemer\\

{\bf{Spørgsmål 22: Ydeligere kommentarer?}}\\
Dette spørgsmål er til at finde eventuelle fejl/overseelser i vores spørgeskema

\section*{Væsentlige resultater}
\subsection*{Målgruppe}
Følgende spørgsmål viser at spørgeskemaet er prinært kommet ud til mænd i alderen 18 - 21 år, som er studerende på videregående uddannelse. Dog er gruppen med lønmodtagere også stor (27 \%) som er den primære målgruppe mht. det automatiserede hjem\\
\figurspg{Figurer/Sporgeskema/Kon.png}{Køn}{kon}{1.0}
\figurspg{Figurer/Sporgeskema/Alder.png}{Alder}{alder}{1.0}
\figurspg{Figurer/Sporgeskema/Beskaftigelse.png}{primære beskæftigelse}{beskaftigelse}{1.0}\clearpage
\subsection*{Informationer om bolig}
Følgende spørgsmål viser hvilke boligtyper personerne der har besvaret spørgeskemaet bor i. Her er det værd at bemærke at 35\% ejer en bolig. Dem der ejer en bolig burde være dem med størst interesse i et automatiseret hjem, pga. de typisk ikke har problemer med at bruge penge på noget man ejer selv. Det er også værd at bemærke at 33\% bor i paracelhus / villa og 47\% bor i lejlighed. Derudover bor de fleste i et hus ældre end 1990 (61\%)\\
\figurspg{Figurer/Sporgeskema/EjerLejer.png}{Ejer/Lejer}{ejerlejer}{1.0}
\figurspg{Figurer/Sporgeskema/Boligtype.png}{Boligtype}{boligtype}{1.0}
\figurspg{Figurer/Sporgeskema/BoligAlder.png}{Alder på bolig}{boligalder}{1.0}\clearpage
\subsection*{Informationer om husstanden}
\figurspg{Figurer/Sporgeskema/PersonerIHusstanden.png}{Antal personer i husstanden}{personerIhusstand}{1.0}
\figurspg{Figurer/Sporgeskema/BoligAntalTimerTom.png}{Antallet af timer boligen står tom}{tomtid}{1.0}
\subsection*{Vaner}
\figurspg{Figurer/Sporgeskema/VanerSluk.png}{Hvilke apperater slukker du for at undgå unødig standbystrøm}{standbystrom}{1.0}
\subsection*{Hovedspørgsmål}
Følgende 2 spørgsmål viser hvor mange personer der har et automatiseret hjem, og hvor mange der ville have interesse - det kan ses at der generelt ikke er mange der ejer et automatiseret hjem, tilgengæld er der rigtig mange med interesse for det\\
\figurspg{Figurer/Sporgeskema/Hjemautomatiseret.png}{Antal personer med automatiserede hjem}{automatiseret}{1.0}
\figurspg{Figurer/Sporgeskema/Interesse.png}{Vil du have interesse for et intelligent system, som automatisk tilpasser dit hjems varme og elektroniske apparater}{interesse}{1.0}\clearpage

\subsection*{Spørgsmål til personer med automatiseret hjem}
\figurspg{Figurer/Sporgeskema/AutProducent.png}{Hvilken/hvilke producent/producenter benyttes til dit automatiserede hjem}{autprod}{1.0}
\figurspg{Figurer/Sporgeskema/AutStyrer.png}{Hvilke funktioner styrer det automatiserede hjem}{autstyrer}{1.0}
\figurspg{Figurer/Sporgeskema/AutFungerer.png}{Hvordan syntes du det automatiserede system fungerer}{autfungerer}{1.0}
\figurspg{Figurer/Sporgeskema/AutFlereTing.png}{Kunne du tænke dig at flere af dine enheder kunne snakke sammen i dit nuværende system}{autflereting}{1.0} \noindent Af releavante kommenetarer til "ja - indtast hvilke" var:\\\\
\indent Alarmsystem\\
\indent Automatisk varmeregulering\\
\indent Automatisk lysregulering\\
\indent Elektriske apperater som eks. tørretrumler og vaskemaskine\\
\indent Integration mellem IHC og Sonos lydsystem\\
\indent Integration mellem TV og lys (eks. automatisk lysdæmpning ved start af film)\\
\indent Styring vha. telefon\\
\indent Automatisk udluftning\\
\indent Start af apperater når elprisen er billig\\
\indent Røg og gasalarmer kommunikerer med systemet

\subsection*{Spørgsmål til personer med interesse i et automatiseret system}
{\bf{De følgende spørgsmål tager udgangspunkt i følgende spørgsmål: Hvilken type automatisering ville du personligt priotere højest, for at en automatiseret løsning ville være relevant for dig}}
\figurspg{Figurer/Sporgeskema/PrioKaffe.png}{Køkkenapperatur som automatisk brygning af kaffe}{priokaffe}{1.0}
\figurspg{Figurer/Sporgeskema/PrioAutoSluk.png}{Automatisk sluk af standbyapperatur}{prioautosluk}{1.0}
\figurspg{Figurer/Sporgeskema/PrioVarmAuto.png}{Varmestyring i form af automatisk regulering}{priovarmauto}{1.0}
\figurspg{Figurer/Sporgeskema/PrioAutoLys.png}{Sensorer i form af automatisk lys}{prioautolys}{1.0}
\figurspg{Figurer/Sporgeskema/PrioAutoGardin.png}{Automatiske gardiner}{prioautogardin}{1.0}

{\bf{I hvor høj grad vil du prioritere følgende fordele ved et automatiseret hus}}
\figurspg{Figurer/Sporgeskema/PrioBespar.png}{Økonomisk besparende}{priobespar}{1.0}
\figurspg{Figurer/Sporgeskema/PrioMiljo.png}{Miljøvenligt}{priomiljo}{1.0}
\figurspg{Figurer/Sporgeskema/PrioTid.png}{Tidsbesparende}{priotid}{1.0}
\figurspg{Figurer/Sporgeskema/PrioSikker.png}{Sikkerhedsforøgende}{priosikker}{1.0}

\figurspg{Figurer/Sporgeskema/Villig10aar.png}{Ville du være villig til at invistere i et automatiseret system hvis det først tjente sig selv hjem efter 10 år?}{villig10aar}{1.0}

\section*{Spørgsmål til personer uden interesse i et automatiseret system}
\figurspg{Figurer/Sporgeskema/hvorforikkeinter.png}{Hvorfor ville et automatiseret hus ikke være interessant?}{hvorforikkeinteressant}{1.0} \noindent Af relevante kommenetarer til "Andet" var:\\\\
\indent Ikke brug for det\\
\indent Fordi huset er tomt på forskellige tidspunkter\\
\indent Mangel på programmerbarhed og konfiguerbarhed med nuværende systemer\\
\indent Vil selv være i kontrol over siturationen

\subsection*{Har du nogle gode ideer til hvad et inteligent system skal kunne gøre?}
Se bilag \ref{sec:bilcd}

\subsection*{Ydeligere kommentarer}
Se bilag \ref{sec:bilcd}

\subsection*{Tilfredshed Zensehome}
\figurspg{Figurer/Sporgeskema/ZensehomeFungerer.png}{Hvordan syntes folk Zensehome fungerer}{zensehomefungerer}{1.0}
Bemærk datagrundlaget er her kun 3 personer

\subsection*{Tilfredshed: IHC}
\figurspg{Figurer/Sporgeskema/IHCFungerer.png}{Hvordan syntes folk IHC fungerer}{ihcfungerer}{1.0}
Bemærk datagrundlaget er her kun 6 personer