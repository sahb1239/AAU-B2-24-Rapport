%Skrevet af Jimmi.
%Sidst rettet: 14-12-2013, 22:00, Jimmi
\label{sec:zensehome_interview}
\section*{Læsevejledning til interview af Zensehome}
Bilaget for interviewet, er herunder vedhæftet. Spørgsmålene er blevet indledt i forskellige katagorier, for at kunne overskueliggøre det mundtlige interview. Nedenstående er det færdiganalyserede data, hvilket betyder, at nedenstående materiale er blevet filtreret ift. relevans. \\

\section*{Interview af Zensehome}
{\bf Købsårsager} \\
\begin{itemize}
    \item Primært nybyg (90 \%) - folk vil gerne have et komplet system når de køber nyt
    \item Energibesparende
    \item Komfort
    \item 1600 - 1700 installationer
    \item 55 000 controllers solgt
\end{itemize}

{\bf Software}
\begin{itemize}
    \item 10.000 linje kode
    \item 4800. 69% 
    \item 2400, 1200 -> nedaf. 93%
    \item Har et retry på 10 gange
    \item PC boks USB kører over COM port
    \item PowerLine
    \item ATMEL processor
\end{itemize}

{\bf Brugerinteraktion}
\begin{itemize}
    \item Du skal selv sætte scenarier op
    \item Ingen mulighed for automatisk at gætte sig til hvordan systemet skal sættes op
    \item Knapperne lyser, når en enhed kontakten er forbundet til er tændt
    \item PC boks logger strømforbrug
    \item PC boksen bruges til at programmere systemet samt scenarier
    \item Mulighed for at slukke når der bruges standby strøm
    \item Simpel brugerflade
\end{itemize}

{\bf Kommende funktioner}
\begin{itemize}
    \item Styring af gardiner (Norge)
    \item Måling af vandforbrug
    \item Måling af varmeforbrug
\end{itemize}

{\bf Fordele}
\begin{itemize}
    \item Slipper for standbystrøm
    \item PC boks benytter kun 0.3 - 1.2 W
    \item Får at vide detaljeret strømforbrug
    \item Fleksibelt og konfigurerbart
    \item Op til 16 actions inde på en stikkontakt
    \item Stikkontakterne benytter kun 0.1 W
    \item Standby-funktioner (se også ulemper)
    \item Vælger tidsbetinget indstilling, så der måles på angivet W efter tid.
    \item Muligheder for på alle enheder at lave actions
    \item Mulighed for at se om udstyr er tændt på anden lokation såvel som på     \item mobilen (Android, iPhone)
    \item Hjemmesimulering
    \item Det er muligt at programmere alle controllers til at være hjemmesimulerende. Dette er en fordel i forbindelse med længerevarende uderejse.
    \item Mulighed for at slukke alt vha. broadcast
    \item Mulighed for at koble flere enheder sammen
    \item Mulighed for at importere plantegninger / jpeg filer
    \item Stor fordel med repeating
    \item 244 outputs og uendeligt antal kontakter
    \item Mulighed for at opgradere og nedgradere firmware
\end{itemize}

{\bf Ulemper}
\begin{itemize}
    \item LED skal skal have meget ampere til opstart
    \item Elektronisk støj
    \item Ingen varmestyring
    \item Ingen logning af adgang
    \item Ingen automatisk standby opsætning
    \item Ingen reelle sikkerhedsbarrierer
    \item Tænd og sluk er kun begrænset af båndbredden
    \item Ingen automatisk identifikation af enheder via ID - du skal manuelt vælge produktnavn
    \item Pris: 495 kr. for en kontakt     (uden rabat)
    \item PC boksen reagerer synkront og ikke asynkront
    \item Boksen har som standard ikke netværksmodul
    \item Opsætning af router - portforwarding - statisk ip
    \item Ikke krypteret
    \item Ikke mulighed for at konfigurere på app’en
    \item Ulempen ved flere stik er at systemet bliver langsomt
\end{itemize}