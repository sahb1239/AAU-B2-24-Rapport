\label{sec:projektforslag}
\section*{Sikkerhed ved pengeoverførslen på mobilen}

Mange nye betalingsmulighed skyder op - noget af det nyeste er nemme pengeoverførsler på ens smartphone, hvor der kun skal bruge modtagerens telefonnummer. Af eksempler kan blandt andet nævnes Danske Banks MobilePay og Swipp, som ejes af en samling af mange danske banker. Systemet har mange fordele da det først og fremmest er nemmere at huske et mobilnummer end et kontonummer, hvilket for nogle folk også føles mindre "privat". Disse overførselsmuligheder har fokus på to problemer, som det løser. Her er eksempler på disse:
\subsubsection*{Hurtige overførsler mellem venner}
Hvis et vennepar er på restaurant og vil dele regningen, vil det tage længere tid og desuden give mere arbejde for restauranten, hvis begges kort skal køres igennem. Her vil det i stedet være muligt at den ene blot lægger ud, hvorefter den anden laver en mobiloverførsle og pengene står på den andens konto med det samme. 

\subsubsection*{Onlinehandel på dba.dk og lignende}
Før i tiden tog en handel et godt stykke tid, hvis man ønskede at købe brugte varer på sider som dba.dk og guloggratis.dk, da det var nødvendigt at vente 1-2 hverdage før pengene kunne ses på modtagerens konto. Dette problem er blevet løst ved hjælp af de hurtige overførsler på smartphones, da modtageren med det samme får bekræftelse på at pengene går ind og denne kan derfor sende varen med det samme. Dette er uanset om afsender og modtager har samme bank.

\subsection*{Problemstillingen}
Vi har mange personlige oplysninger på vores smartphones - netbank, e-boks og disse hurtige overførsler. Men hvor meget vægt bliver der egentligt lagt på sikkerheden? De fleste ved at der skal være installeret antivirus, firewalls og lignende på computeren, men har man den samme tankegang når det kommer til ens smartphone? Hvad kan der gøres for at opnå højere sikkerhed?