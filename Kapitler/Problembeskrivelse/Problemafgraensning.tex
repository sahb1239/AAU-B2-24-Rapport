%Skrevet af Jimmi.
%Sidst rettet: 19-11-2013, 16:34, Jimmi
Problemanalysen har belyst konkrete samfundsrelaterede problemstillinger ved det automatiserede hus. Ud over at belyse problemstillingerne forbundet med de eksisterende teknologier, har problemanalysen også givet et solidt udgangspunkt hen imod relevante samfundskritiske problemstillinger. De eksisterende teknologier, er blevet dissekeret; hvor især ulemperne fra teknologierne kan revideres – i håbet om at – i sidste ende, at konstruere en mere raffineret teknologi. \\\\
Dataindsamlingen-afsnittet (Afsnit \ref{sec:dataindsamling-afsnit}) fremhæver særligt de økonomiske besparelser, som kan opnås ved én automatiseringsløsning, men lige så vigtigt, er brugernes ønske om et tidsbesparende system. Nutidens automatiseringsløsninger fokuserer primært på de økonomiske besparelser, simpelthen ved at slukke for de elektroniske apparaturer ved unødig brug. Zensehome’s automatiseringsteknologi blev grundlagt i bestræbelserne på at opnå økonomiske besparelser. Foruden de økonomiske besparelser, blev systemet konstrueret således at systemet kunne implementeres på det gamle el-net. Blev husstanden på et senere tidspunkt udbygget, var det essentielt, at forbrugerne selv kunne tilpasse systemet. \\
I bestræbelserne på at optimere systemet yderligere, blev der på baggrund af dataindsamlingen (Afsnit \ref{sec:dataindsamling-afsnit}), lagt særlig vægt på forbrugernes ønske: tidsbesparelse. Under interviewet (Afsnit \ref{sec:zensehomeinterview}) med automatiseringen-producenten, Zensehome, blev der bemærket, at alt interaktion mellem forbrugeren og huset, skulle foregå via tablets, smartphone eller computer. Ydermere krævede afspilningen af scenarierne, at enheden skulle findes frem – applikationen skulle åbnes – for til sidst at trykke på det ønskede scenarie. Processen kunne være tidskrævende og anstrengende for forbrugeren. På baggrund af respondenternes høje ønsker om et mere tidsbesparende system, konkluderede gruppen, at det er her fokus skulle ligges. I præcis samme boldgade var stemmestyring-controllere, som viste et højt tilfredshedsniveau iblandt forbrugerne. Henholdsvis 55\% og 36\% af Siris kunder viste fuldt tilfredshed, samt delvist tilfredshed (Afsnit \ref{sec:infosearch}). Ydermere kunne det konkluderes, at teknologien havde fået et gennembrud blandt verdens største elektronik-fabrikanter – Samsung, Android-styresystemet samt IPhone (Afsnit \ref{sec:stemmestyring}).
