\label{sec:zensehome}
Zensehome er som tidligere nævnt en virksomhed, der har udviklet et produkt til automatisering af huse. Produktet automatiserer en hel del elkomponenter tilsluttet el-nettet.
Systemet hjælper med at formindske strømforbruget, men hjælper også med at gøre tingene i hjemmet lettere for brugeren. \\
Systemet fungerer på den måde, at stikkontakter, lampeudtag udskiftes med Zensehomes kontakter, den såkaldte controller. Derefter bliver disse enheder opkoblet vha. det som Zensehome kalder en PC-boks. Denne pc-boks fungerer ved, at den modtager og sender datapakken til remote-enheder. Remote-enheder kan eksempelvis være lamper, sensorer, EL-gulvvarme og stikkontakter.
\subsubsection{Remote enheder}
Zensehomes systems kan maksimalt håndtere 244 remotes. Denne begrænsning eksisterer, da løsningen kører på en meget langsom overførelseshastighed. Dette medfører, at systemet højest kan sende og modtage til en hvis mængde remotes.
Systemet kører med 2.400 bit i sekundet. Dette er meget langsomt, hvis dette skulle perspektiveres til internettet - eksempelvis computere og lignende.
Firmaet har valgt at bruge denne hastighed ud fra flere tests, som de har foretaget. Testene indebar, at der blev sat en modtager boks op i den ene ende af huset samt en sender i den modsatte ende af huset. 
Derefter udsender sender boksen en række signaler til modtager boksen. Signalerne bliver udsendt i en prædefineret antal og derfor ved test-devicen hvor mange signaler modtager-boksen skal modtage.
Zensehomes testresultater viste, at den største succesrate, kunne opnås ved 2.400 bit/s. Her er der en modtagelsesrate på 93\%, hvilket Zensehome mente at være acceptabelt. Ved interviewet med Zensehome\ref{sec:zensehomeinterview} blev der ligeledes spurgt, hvad der ville ske, hvis de sendte ved 4.800 bit/s. Svaret var, at dette ville resultere i en forringet modtagelsesrate på 63\%. Udover dette nævnte Zensehome at modtagelsesraten ikke blev højere af at sænke overførelseshastigheden til 1.200 bit/s
\subsubsection{Hvordan virker en enhed (remote)} En remote virker ved, at den modtager et input i form af en pakke, som udsendes af kontakten, eller ved at forbrugeren får remoten til at lave et output.  Hvis forbrugeren fra mobil-applikationen eller PC-boksens software vælger at slukke lyset i stuen, vil PC-boksen sende et output, der går ind i den pågældende remote. Derefter venter remoten i maks 2 sekunder på, at den modtager et svar om, at den har udført den handling, som den blev bedt om. Hvis remoten ikke modtager et svar tilbage vil den forsøge at sende datapakken igen.\\ \figurw{Figurer/kontakt.png}{Remote for kontakt \figuregroup}{Remote}{1.0} På Figur (\ref{fig:Remote}) ses en kontakt med fire funktioner og to elektroniske apparater. De to elektroniske apparater er i form af en lampe og en stikkontakt til TV´et, som hver især har 2 funktioner. 
Funktionerne som vist på Figur (\ref{fig:Remote}) kan have forskellige navne (eller ID). Dette betyder at selvom man trykker på kontakt1´s funktion 1, kan den oversætte dette til, at hvis den skal kontakte lampe1 med funktion 2, da dette er den ønskede handling på lampen.
Dette er en form for synonym for, hvad man skal trykke på således at Kontakt1 (Funktion1) ---> Lampe1 (funktion2).
Hvis forbrugeren trykker på kontakten (kontakt 1 med et kort tryk) ville den sende et signal til den elektroniske genstand, som i dette tilfælde ville være lampe 1 svarende til funktion 1, som illusteret på Figur (\ref{fig:Remote}). Lampe 1 ville derefter modtage signalet og efterfølgende ændre status for hvilke funktioner, som lampen har. Det betyder, at hvis lampen er tændt, vil den slukke og hvis den er slukket, tænder den. Der er dog eksempler på, at den elektroniske genstand har flere funktioner end blot to funktioner, nemlig tænd og sluk. Eksempelvis kan der være funktioner som lysdæmpning, standby slukning og sensorstyring. I dette tilfælde vil kontakten, som forbrugeren trykker på, sende et id samt en funktion med i signalet, som repræsenterer funktionen, som remoten skal gennemføre.\\ Dette betyder, at en kontakt kan have mere end en funktion som vist på Figur (\ref{fig:Remote}), da den måler input på flere forskellige måder. Dette er f.eks. lange og korte tryk, hvor man kan sætte lysdæmpning til et langt tryk (funktion 3 på fig. \ref{fig:RemoteTilLampe}) og korte tryk til at slukke (funktion 1\&2 på Figur (\ref{fig:Remote})). Forbrugeren kan selv vælge en handling, som kontakten skal udføre, når den opfanger et langt eller kort tryk. Et scenarie kunne eksempelvis opstilles på den måde, at en bruger ønsker, at foretage et kort tryk på kontakten for at tænde lampen. Herefter kunne brugeren have behov for at dæmpe lyset, hvilket så bliver gjort ved et langt tryk på samme knap. \figurw{Figurer/remotetillampe.png}{Remote (kontakt) til lampe \figuregroup}{RemoteTilLampe}{0.5} På Figur (\ref{fig:RemoteTilLampe}) ses det, at forbrugeren trykker på en knap på kontakten, som derefter sender et signal til lampen. Dette signal indeholder et id. Signalet modtages så af lampen, som derefter ser efter, hvilket id som blev sendt til lampen. Herefter udføres den valgte handling.
For at give et overblik over systemet har hver en handling et id, som gør, at der kan tilknyttes flere enheder og handlinger til en kontakt. Dette giver mulighed for at have en funktion til at slukke alt, hvad der ønskes i hele huset på en gang. Funktionen sluk alt fungerer via. broadcast\footnote{Udsending af datapakker, til flere enheder på en gang}, hvor man har en enhed i form af en kontakt, som sender et signal til flere elektroniske enheder. Hvis en af enhederne fejler, vil denne genstand ikke slukkes, da genstanden ved broadcast ikke har en sikkerhed for, at signalet bliver modtaget i enheden. \\

Den grafiske illustration af dette, ville se således ud:

\figurw{Figurer/flereremotes.png}{Fra 2 remotes til flere komponenter \figuregroup}{flereremotes}{1.0} 

\subsubsection{Zensehome App}
Det er muligt at tilslutte en installation af Zensehome til internettet, hvis man tilkøber en PC-boks med netmodul. Modulet har en app, som fungerer både på styresystemet Android og iPhones iOS-styrestem. Dette giver forbrugeren mulighed for, at kunne styre et system, selvom brugeren ikke er til stede fysisk. Dette gøres enten via TCP protokollen, der kan sikre, at pakken bliver modtaget eller af UDP protokollen, som ikke har nogen sikkerhed på, at pakken bliver modtaget. UDP protokollen er nyttigt i broadcast, da den dermed ikke skal kontrollere, at signalet blev succesfuldt modtaget, hvilket øger hastigheden. Signalet modtager applikationen via routeren, som skal lukke dataen ud gennem det interne net i hjemmet. For at forbrugeren kan tilgå systemet via applikationen fra internettet, skal systemet have adgang til internettet. Dette kræver dog, at routeren er sat op til portforwarde, da det normalt ikke er tilladt, at udefrakommende kan kommunikere med systemet.

\figurw{Figurer/app.png}{Pakkers vej igennem systemet via APP/PC-Boks \figuregroup}{app}{0.7}
På Figur (\ref{fig:app}) ses en trådløs router, som har 3 elektroniske komponenter forbundet. Ikke alle disse genstande kan tilgås fra internettet - derfor bruger man port forwarding. Portforwarding giver pakkerne direktioner til, hvordan pakkerne skal komme igennem netværket. Dette er illustreret på Figur (\ref{fig:app}), der viser, hvordan en pakke får forbindelse til den rigtige "computer/elektronik udstyr".\\ PC-boksen kører over port 10001. Dette er dermed standardporten, som PC-boksen kan kontaktes på. Hvis det ønskes, kan brugeren ændre dette for at forøge sikkerheden. Dermed opsættes routeren til at sende alt data, som kommer ind på port 10001 videre til PC-boksen, som vist på figur (\ref{fig:app}). Denne konfiguration gør systemet let tilgængeligt, men også mere usikkert på nogle områder da det giver endnu en indgang til systemet udefra, som kan misbruges.