%Skrevet af Mikkel og Kim.
%Sidst rettet: 15-12-2013, 18:50, Jimmi
\label{sec:Interessant}
I dette afsnit undersøges de forskellige interessenter. Deres subjektive påvirkninger analyseres, for at belyse de positive, såvel som de negative følger af et stemmestyret system. Interessenterne undersøges med fokus på hverdagen, hvoraf de rutineprægede hverdagsopgaver fremføres. Interessenternes påvirkninger vil ene og alene blive set fra den indgangsvinkel, at de allerede har et automatiseret system installeret. Det betyder, at interessenternes påvirkninger vil blive set uafhængigt af nuværende automatiseringssystem, men ene og alene ud fra et supplerende perspektiv, som et stemmestyret system kunne bidrage til. \\

Hver interessent vil blive gennemgået hver for sig, som afslutningsvis vil munde ud i et afgrænsningsafsnit, som vil samle, og opretholde interessenternes påvirkninger.

\section{Gennemsnitsfamilien} 
For den gennemsnitlige danske familie, vil et stemmestyringssystem kunne bidrage til en mere effektiv såvel som komfortabel hverdag. De tidsbesparende elementer kunne her være, at samtlige controllere, scenarier og konfigurationer kan foretages vha. simple stemme-kommandoer. Et praktisk anvendelseseksempel kunne være konsekvente handlinger, såsom at slukke alt lyset i stuen, i gangen og fjernsynet, blot ved at sige en simpel kommando. Under aftensmaden vil forældrene hurtigt kunne foretage en kontrol, om hvorvidt børnene har husket at slukke lyset på deres værelser. Dette ledsager ligeledes til øget komfortabilitet, idet at brugerne vil kun foretage handlinger, uden at skulle foretage fysiske bevægelser. Et praktisk eksempel kunne her være, at brugeren har 3 lamper i stuen, som ønskes slukket i forbindelse med film-aften. Her kunne brugerne – i stedet for at rejse sig, og slukke dem manuelt – blot sige en kommando, for at udføre en sluk-handling. \\

Et stemmestyringssystem vil dog ikke kun have positive følger. De negative påvirkninger kunne være, at den enkelte familie, opstiller en række scenarier, som går ud over det sociale sammenværd. Dette kunne eksempelvis gå ud over børnene, som måske tidligere er blev lagt i seng sammen med godnatlæsning af forældrene. Her kunne stemmestyring lede til, at forældrene blot anvender en kommando, som slukker alt elektronisk apparatur på børneværelserne, for at indikere at det er sengetid. Foruden dette kunne stemmestyringssystemet lede til et Big Brother-kontrollerende hjem, hvor forældrene hurtigt kan få en statusopdatering på børnenes tændte apparaturer, for derved at undersøge om børnene overholder sengetiderne.
De førnævnte eksempler er blot en bestanddel af praktiske anvendelsesmuligheder stemmestyring kan bidrage til.

\section{Handicappede og ældre} 
Et andet segment i forhold til de private brugere, er de ældre og de handicappede. For dem ville et sådant system ikke kun øge komforten, men især livskvaliteten. Systemet vil være at betragte som et hjælpemiddel, der hjælper til øget funktionalitet i dagligdagen.\\Et eksempel kunne være at kørestolbrugere vil kunne bede systemet om at åbne døren, så det bliver nemmere at komme rundt i hjemmet.\\
I forhold til de ældre, vil systemet også kunne klare nogle af deres besværligheder i hverdagen - dette kan dog have nogle negative følger. Stemmestyringens kommandoer kræver at brugeren kan huske kommandoerne, hvilket kan være problematisk for den ældre. \\ 
Lægges der i stedet vægt på ét stemmestyringssystem i hjemmeplejen, er der ligeledes både positive og negative påvirkninger. På sigt vil det føre til økonomiske besparelser, idet at borgerne kan foretage flere handlinger på egen hånd, og derved vil plejehjemmet kunne spare personale. Fokuseres der på borgerne, vil dette betyde mindre socialt samvær i forhold til en hjemmehjælper, som det sås da robotstøvsugeren kom ud i hjemmeplejen \cite{robotstovsuger}.

\section{Virksomheder}
Effektivisering er altid i fokus på erhvervsdrevne virksomheder, hvorved en stemmestyret løsning kunne bevirke til øget produktivitet. Et praktisk eksempel kunne være lagerstyring, som ofte kræver konstant ajourføring på opmagasinering. Herforuden ville systemet kunne bidrage til en lang række forbedringer, såsom at informere brugeren, om hvor specifikke vare er lokaliseret fysisk på lageret. Dette er et tidsbesparende element, som ville lede til øget effektivitet, idet at medarbejderne ikke længere skal bruge tiden forgæves på at finde de ønskede varer. Et firma ved navn "Millarco"\cite{Int-virksomheder} har fået en stemmestyringsløsning implementeret, som øger deres effektivitet på lageret. Her benytter de stemmestyringssystemet til blandt andet, at få en lagerstatus samt varebestillinger. Det negative kunne være, at stemmestyringsløsningen automatiserer flere processer, som før i tiden krævede en ekstra medarbejder. På denne måde kan systemet bivirke til fyringer. \\

For virksomheder er sikkerhed alfaomega. Server-rum og andre aflåste lokaler vil kunne tilgås ved stemmemanipulation, hvorved medarbejdere eller andre udefrakommende personer kan tilgå følsomme oplysninger. Ydermere vil system-shutdowns, lede til massive produktivitetstab, afhængig af hvordan den enkelte virksomhed har valgt at sikre sig. 

\section{Offentlige institutioner} 
Offentlige institutioner som børnehaver, plejehjem og biblioteker vil bruge et stemmestyret system, som supplement til deres daglige gøremål.\\ 

En børnehave vil kunne bruge et stemmestyringssystem til at holde børnene aktive hele dagen. Dette kan lade sig gøre, ved at der bliver udviklet et system som har nogle forskellige aktiviteter, som børnene ville kunne styre med stemmen. Dette kan resultere i at børnene på et tidligt stadie, lære at bruge IT i deres dagligdag. \\

På bibliotekerne ville man kunne bruge systemet til hurtigt at finde de bøger man leder efter. Bibliotekarerne ville ligeledes kunne ajourføre bibliotekernes lager ved brug af stemmen. Dette effektiviserer bibliotekerne i en retning, som både de ansatte kan få glæde af, men også forbrugerne, som hurtigt vil kunne finde deres ønskede materiale. Systemet vil have en række konsekvenser, som at forbrugerne ikke længere vil kunne stille ukonkretiserede spørgsmål, som ikke er inkorporeret i systemet. Dette vil stadig kræve, at bibliotekarer er fysisk tilstede. Systemet vil dog til stadighed – vil kunne løse mange generelle problemstillinger.

\subsection{Afgrænsning af interessenterne}
Efter at have belyst de forskellige, vigtige interessenter ses det, at der er utrolig mange aspekter i et stemmestyret system. Der er blevet fremvist en lang række tidsmæssige besvarelser, såvel som forøget komfort, hvilket er generelt for alle interessenterne. Ydermere kunne der berettes, at stemmestyring har negative følger, som sikkerhedsmæssige aspekter samt formindskende socialt samvær. For at bygge videre på Zensehomes løsning, er der blevet afgrænset til, at fokus skal ligges på gennemsnitsfamilien. Ligeledes ønsker gruppen at se bort fra de sikkerhedsmæssige aspekter, som er stærkt associeret med virksomheder og ældreplejen. Ved den gennemsnitlige familie kunne det desforuden være interessant at kigge videre på forøget tidsbesparelse såvel som komfort. Nedenunder er opstillet de positive og negative påvirkninger den gennemsnitlige familie vil kunne opleve ved et stemmestyret system: 
\begin{itemize}
    \item Positive påvirkninger
    \begin{itemize}
        \item Tidsbesparelse og øget komfort
        \begin{itemize}
            \item Styring af controlleres status - dvs. at tænde og slukke for elektriske apparater
            \item Opsætning, redigering samt sletning af scenarier og controllers
            \item Økonomiske besparelser (hurtig kontrol/ændring af controllers status)
            \item Øge fleksibilitet i hverdagen
        \end{itemize}
    \end{itemize}
    \item Negative påvirkninger
    \begin{itemize}
        \item Formindske det sociale samvær
        \begin{itemize}
            \item Forældrenes kvalitetstid med især børnene kan formindskes
            \item BigBrother-lignende hjem, hvor overvågning tager overhånd
        \end{itemize}
    \end{itemize}
\end{itemize}

Valget af interessenterne er ydermere baseret ud fra resultaterne af spørgeskemaet \ref{sec:sporgeskema-afsnit}, som hovedsageligt henvendte sig til den almindelige dansker. Gennemsnitsfamilien er derfor valgt som interessant.