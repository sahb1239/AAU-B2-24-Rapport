%Skrevet af Jimmi.
%Sidst rettet: 14-12-2013, 22:00, Jimmi
\label{sec:zensehomeinterview}
Zensehome er én af producenterne af automatiseringscontrollers til det automatiserede hjem, som holder til i Nørresundby. For at høre mere omkring systemet, blev der taget telefonisk kontakt til Zensehome. Zensehome tilbød i denne forbindelse, at gruppen kunne få systemet demonstreret, som gruppen takkede ja til. Formålet ved denne demonstration, var at få en udvidet indsigt i, hvordan et automatiseringssystem fungere - både det rent opsætningsmæssige, såvel som det softwaremæssige. \\

Gruppen forbedredte inden demonstrationen en række kvalitative spørgsmål, som havde til formål, at belyse købsårsager, teknologiske informationer, såvel som bevidste mangler eller fremtidige planer. \\

Disse kvalitative spørgsmål blev opbygget i henhold til Steinar Kvales\cite{BOOK_KVALE}\footnote{Steinar Kvale er leder for Center for Kvalitativ Metodeudvikling sammesteds. Kvale anses som en autoritet inden for kvalitativ forskning.} 7 stadier i en interviewundersøge.
\begin{enumerate}
    \item Tematisering. Her formulerede gruppen indledningsvis en række spørgsmål, som tog udgangspunkt i det, som gruppen ønsker at finde ud af
    \item Design. Her gennemtænkes alle besvarelsesmuligheder, for at undersøge om de er relevante for tematiseringen
    \item Interview. Spørgsmålene blev stillet i en rækkefølge, så de mest relevante spørgsmål blev stillet først. Ydermere var gruppen på forhånd klar over, at demonstrationen ville blive afholdt af en sælger. Sælgerens dybere teknologiske viden blev overvejet, og spørgsmålene blev opstillet, så de var tilpasset til interviewerens kompetenceniveau.
    \item Transskribering og Analyse. Demonstrationen og interviewet nu færdig, og alle besvarelser skulle nu analyseres. Dette betød at det mundtlige materiale, blev finskrevet og kategoriseret.
    \item Verificering. Her diskuterede gruppen, interviewerens pålidelighed og validitet. Det kunne f.eks. være var specifikke problemer forbundet med systemet, som sælgeren bevidst forsøgte at drosle ned, eller bortargumentere.
    \item Rapportering. Sidste stadie er omdannelsen fra det analysebehandlede materiale til en rapportform, hvorved de relevante perspektiver fremføres.
\end{enumerate}

Efter interviewet, kunne gruppen hermed opstille en række fordele og ulemper ved systemet (herunder et udkast) (se hele interviewet i \ref{sec:zensehome_interview}: 
\begin{itemize}
    \item Fordele
    \begin{itemize}
        \item Brugeren slipper for standbystrøm, ved at systemet kan opsættes i tidsintervaller, hvorved elektroniske apparaturer slukkes automatisk
        \item Mulighed for at benytte det gamle el-net under implementationen
        \item Huset kan tilgås uanset enhed og lokation. Køre på internettet, hvorved brugeren kan styre og konfigurere sit system eksternt.
    \end{itemize}
    \item Ulemper
    \begin{itemize}
        \item Systemet kan ikke køre sammen med varmestyringen og alarmsystemet
        \item Prisen er forholdsvis høj (495,- / kontakt)
        \item Elektronisk støj som kan betyde i visse tilfælde, at lyspærer af ukendte mærker - ikke er understøttet
        \item Controllers\footnote{Zensehomes komponenter, såsom stikkontakter, lampeudtag og sensorer} og scenarier\footnote{En konfiguration som afspiller en række handlinger} kan kun afspilles igennem enheder, såsom smartphone, computeren, eller tablet. Dette går ud over det tidsbesparelsen i systemet.
    \end{itemize}
\end{itemize}

Gruppen ønskede ligeledes at finde ud af, hvor stor interesse danskerne har for et automatiseret hjem. Udbredelsen af systemet blev estimeret af Zensehome til at være omkring 1600 - 1700 installationer, primært i private huse. Derudover blev det oplyst at 90 \% af installationerne var i nybyg.\\

Zensehome er specielt relevant i dette projekt, da deres produkt der er let at implementere i gamle huse, idet at enhederne kommunikere over el-nettet (Afsnit \ref{sec:tekzensehome}). Det kan ud fra mødet konkluderes, at tidsbesparelse kunne være ét af elementerne til en problemstilling, som der kunne fokuseres på.
