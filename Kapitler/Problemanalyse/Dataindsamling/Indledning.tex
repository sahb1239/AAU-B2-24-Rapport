%Skrevet af Jimmi.
%Sidst rettet: 18-11-2013, 21:43, Jimmi
\label{sec:dataindsamling-afsnit}
\textit{Dataindsamling er en yderst effektiv arbejdsmetode, som kan anvendes såfremt én eller flere samfundsrelaterede problemstillinger ønskes belyst. Før at dataindsamlingsprocessen kan påbegyndes, bør det overordnede formål stå krystalklart. Dette er vigtigt når det indsamlede data senere skal behandles og analyseres – her bør resultaterne gerne afspejle konkretiserede problemstillinger, forekomster eller andre samfundsrelaterede fænomener. For at lette bearbejdningsprocessen, er der nogle specifikke fremgangsmetoder, som kan anvendes; triangulering, den kvalitative samt den kvantitative metode. De nedenstående arbejdsmetoder anvendes senere til at konstruere et spørgeskemaundersøgelse samt dybdeinterview (Afsnit \ref{sec:sporgeskema-afsnit})}.\\\\
{\bf Metodetriangulering} \\
For at kunne sikre sig et højt empirisk datasæt\footnote{Udgangspunktet for analysen}, blev der anvendt en metodetriangulering, som består af at kombinere flere metoder, for derved at kunne krydsrevidere det indsamlede data. 
Her anvendte gruppen en kvantitativ analysemetode til at belyse eventuelle problemstillinger eller mangler forbundet med automatisering af huset. 
Dette blev gjort i form af en spørgeskemaundersøgelse. Det indsamlede data blev i analysefasen opretholdt med en kvalitativ analyseform; dybdeinterview (\ref{sec:zensehomeinterview}), som blev anvendt til at be- eller afkræfte de kvantitative problemstillinger. 
Metodetriangulering gav en fordybende grundanalysering, som sikrede kvalificeret indsigt i problemernes omfang. Dette dannede et solidt datagrundlag for den videre bearbejdning af rapporten \cite{AnalyseDanmark}. \\

{\bf Den kvalitative metode}\\
\label{sec:kvalitative_metode}
Den kvalitative metode har en lang række fordele, som kan forbedre det planlagte interview eller samtale. Metoden anvendes primært til situationer, hvor konkrete emner eller problemstillinger er uklare, og som ønskes undersøgt i en særlig høj grad. Her kan interaktionen mellem respondenten og intervieweren give mulighed for øget fleksibilitet, så spørgsmålene løbende kan tilpasses efter respondentens svar. Dette føre til en mere dybdegående analyse – som kan være med til afdække manifeste og latente holdninger, såvel som interessante motiver \cite{KommunikationItA}. \\

{\bf Den kvantitative metode}\\
\label{sec:kvantitative_metode}
Den kvantitative metode anvendes i situationer, hvor en konkret tendens ønskes undersøgt. Metoden tager udgangspunkt i mange besvarelser, som ofte opnås igennem et spørgeskema. Det indsamlede data kan segmenteres ud fra nogle overordnede forudsætninger, såsom køn, alder m.v., for derved at kunne anskueliggøre problemet ud fra vidt forskellige samfundslag. Det er dog essentielt, at der er et højt antal besvarelser, for at kunne udlede et konkret samfundsmæssigt problem. Et højt antal besvarelser i spørgeskemaet kan opnås ved at konstruere et professionelt og objektivt spørgeskema, hvor spørgsmålene er udført med stor omhyggelighed \cite{KommunikationItA}. \\