%Dataindsamling
%Sidst revideret af Mads d. 18 Nov kl.13:42
\label{sec:sporgeskema-afsnit}

Der er blevet udarbejdet et spørgeskema for at fastslå om der er nogle problemer eller mangler indenfor de eksisterende automatiseringsløsninger til huse. Dette data, kan bruges til at belyse potentielle behov, eller ønsker - for derved at kunne udarbejde et konkret forbedringsløsningsforslag til de eksisterende automatiseringsløsninger.\\ 
Før at spørgsmålene til spørgeskemaet konstruerers, er der dog nogle kvantitative spørgsmål, som bør overvejes. Her er opstillet nogle eksempler:

\begin{itemize}
    \item Hvem skal spørgeskemaet henvende sig til?
    \item Hvad er formålet med spørgeskemaet?
    \item Hvordan opstilles spørgsmålene objektivt, for at undgå indirekte påvirkning af respondentens svar?
    \item Hvordan kan spørgsmålene formuleres og opstilles mest hensigtsmæssigt, for at undgå fejlkilder? 
\end{itemize}
    
Disse ovenstående eksempler, vil blive uddybet nærmere i afsnit \ref{sec:undersoegelsesdesign}.

\subsubsection{Population og distributering}
For at de førnævnte kvantitative metoder (Afsnit \ref{sec:sporgeskema-afsnit}) kan anvendes, er det vigtigt, at få et højt antal besvarelser. Til dette blev gruppemedlemmernes personlige Facebook-konto anvendt som distributionsmedie. \\

I henhold til spørgeskemaets formål, var det vigtigt, at spørgeskemaet rammede respondenter, som allerede ejede et automatiseret system. For at dette kunne lade sig gøre, var det essentielt at spørgeskemaet blev publiceret igennem andre medier, end blot Facebook. Dette skyldes, at gruppemedlemmernes Facebook-vennekreds, har en alder, som formentlig ikke ejer en automatiseret bolig. \\

For at ramme dette segment, blev spørgeskemaet distribueret igennem relevante teknologifora som livingsmart.dk, hifi4all.dk, eksperten.dk samt andre relevante hobbyforumer. Ved at anvende disse medier opnåede gruppen 272 komplette besvarelser, mens der blev registreret 40 ufuldendte besvarelser.
\subsubsection{Undersøgelsesdesign} 
\label{sec:undersoegelsesdesign}
Spørgeskemaet (Bilag \ref{sec:sporgeskema}) er som udgangspunkt udarbejdet med henblik på at opstille korte og præcise spørgsmål. For at realisere dette, er det vigtigt, at spørgsmålene bliver konstrueret således, at respondenten hurtigt kan tage stilling til de angivne valgmuligheder. Ved at opstille Ja/Nej, såvel som simple punkt-spørgsmål, opnås dette mest hensigtsmæssigt. Spørgeskemaet er desuden konstrueret ved hjælp af SurveyExact.\\

På baggrund af, hvad den pågældende respondent besvarer i sine valgmuligheder, vil de næstkommende spørgsmål være tilpasset efter respondentens forrige besvarelser. Denne funktion kaldes Aktiverings-regel\footnote{Anvendes i analyseværktøjet: SurveyXact}. Spørgeskemaet har 9 aktiveringsregler. Dette er for at konstruere et spørgeskema, som både har relevans for respondenter med et automatiseret hjem, men ligeledes for dem som ikke har. Denne funktion er både til glæde for respondenten, men også for at forebygge fejlkilder, som et spørgeskema, uden disse aktiveringsregler, kunne ledsage til. \\

I 6 ud af 22 spørgsmål, er spørgeskemaet udbygget med en ”Andet”-valgmulighed. Dette er for at forebygge eventuelle fejlbesvarelser, som kunne være ledsaget af, at respondenten ikke har den korrekte valgmulighed til rådighed, og derved vælger den tilnærmelsesvise mulighed. \\

Der er anvendt en kvantitativ metode til at undersøge simple parametre som alder, køn o. lign. i form af 19 spørgsmål, der er målbare statistiske. \\

Flere af spørgsmålene er såkaldte udspecificerende spørgsmål, hvilket er den kvalitative metodeform. Disse spørgsmål har til formål, at få et indblik i, hvilke krav respondenterne har til et automatiseret hjem, uanset om de ejer ét eller ej. Ydermere har disse spørgsmål til formål, at få et samlet indblik i nogle af de funktioner, som respondenterne med et automatiset hjem, efterspørger. \\

\subsubsection{Analysering af data}
\label{sec:analysedata}
Ud fra de 272 fulde besvarelser af spørgeskemaet kan følgende tildens ses (Bilag \ref{sec:sporgeskema} for alle resultater og figurer):
\begin{itemize}
    \item Størstedelen af respondenterne er af det mandlige køn (79\%. Figur \ref{spg:kon})
    \item De er relativt unge respondenter. Mellem 18 - 28 år (72\%. Figur \ref{spg:alder})
    \item Studerende på videregående uddannelse (58\%. Figur \ref{spg:beskaftigelse})
    \item Boligen er i gennemsnit tom i 7 - 9 timer (47\%) og 4 - 6 timer (34\%) (Figur \ref{spg:tomtid})
    \item Den gennemsnitlige respondent slukker ikke for standbyapparater (Figur \ref{spg:standbystrom})
    \item De fleste respondenter ville have interesse i et automatiseret hus (78\%. Figur \ref{spg:interesse})
    \item De højest prioriterede funktioner i et automatiseret hjem er
    \begin{itemize}
        \item Automatisk slukning af standbyapparater og automatisk varmestyring
        \item Systemet skal være økonomisk besparende (71\% i høj grad. Figur \ref{spg:prioautosluk}, \ref{spg:priovarmauto} og \ref{spg:priobespar})
    \end{itemize}
\end{itemize}
Ydermere kan det ses på spørgeskemaet, at LKs\footnote{IHC er en automatiserinsløsning udviklet af Lauritz Knudsen} kunder er delvist tilfredse med deres IHC-løsning (Figur \ref{spg:ihcfungerer}) og at Zensehomes\footnote{Automatiseringsproducent. Denne producent uddybes i afsnittet \ref{sec:zensehomeinterview}} kunder er meget tilfredse med deres løsning (Figur \ref{spg:zensehomefungerer}). \\ På grund af Zensehomes gode ry, har gruppen valgt at sætte et eksklusivt interview op med Zensehome, for potentielt at bruge Zensehomes løsning som hardware forudsætning for vores produkt. \\

Spørgeskemaetundersøgelsen viste ydermere, at respondenterne syntes følgende problemer var relevante:
\begin{itemize}
    \item Energispild ved standby strøm
    \item Manglende stemmestyring af automatiserede hjem
\end{itemize}
Disse problemstillinger er meget relevante i forhold til projektet. De vil her blive gennemgået en efter en.\\

{\bf Energispild ved standby strøm} \\
I Figur \ref{spg:standbystrom} kan det ses at kun ca. 50\% af respondenterne afbryder strømmen til elektriske apparater, ved at slukke på stikkontakten når de forlader hjemmet. På figur \ref{fig:priotid} har 71\% af respondenterne (af dem som var interesseret i automatiserede hjem) svaret at de ville prioritere økonomiske besparelser meget højt, hvis de skulle have et automatiseret hjem.\\
\figurw{Figurer/Sporgeskema/PrioBespar.png}{Hvor højt økonomisk besparelse prioriteres ved stemmestyring af hjem. (213 besvarelser med aktiveringsnøgle)}{priobespar}{1}

Energispild er et mangfoldigt emne, som der er lavet et utal af undersøgelser på. Én af disse undersøgelser er foretaget af Vestforsyningen A/S\footnote{Elforsyning lokaliseret i Holstebro}, som har publiceret pjecen ”Gode Elvaner”\cite{GodeElvaner}. Her kan det ses, at flere apparater ikke bliver slukket og derved har unødig standbyforbrug - faktisk op til 400 kWh om året. Dette er blandt andet routere, computere, lys og tv som er de store syndere. Dette kan bekræftes i forhold til gruppens dataindsamling, som viste at ca. 50\% (Figur \ref{spg:standbystrom}) slukker for deres elektroniske apparater. På grund af dette vælger gruppen at kigge videre på Zensehome, som har specieludviklet software der kan eliminere unødig standbyforbrug. \\

{\bf Manglende stemmestyring af automatiserede hjem} \\
Det kan ses på figur (Bilag \ref{fig:priotid}) at 73\% af respondenterne prioriterer tidsbesparelse højt eller meget højt. Det blev foreslået i spørgeskemaet at stemmestyring er en mangel i automatiserede hjem. På figur (Bilag \ref{spg:interesse}) kan det ses at 78\% af respondenterne ville være interesseret i et automatiseret hjem. I afsnit \ref{sec:infosearch} vil der blive undersøgt om denne problemstilling kan løses. \\
\figurw{Figurer/Sporgeskema/PrioTid.png}{Hvor højt tidsbesparelse prioriteres ved stemmestyring af hjem. (213 besvarelser med aktiveringsnøgle)}{priotid}{1}