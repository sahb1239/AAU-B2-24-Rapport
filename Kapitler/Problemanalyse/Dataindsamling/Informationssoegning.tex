\label{sec:infosearch}
%Led eventuelt efter flere undersøgelser der kan opretholdes%
I dette afsnit fokuseres der på teknologien stemmestyring. Dette skyldes at gruppen tidligere kunne (afsnit \ref{sec:analysedata}) konkludere, at der kunne være et aktuelt behov for et stemmestyret system i et automatiseret hjem. For at afgøre om stemmestyringsteknologien er udviklet i en sådan grad, at den vil kunne fungere i samspil med eksisterende automatiseringsløsninger, undersøges teknologien yderligere i afsnittet \ref{sec:stemmestyring}. \\

Der findes flere færdigudviklede stemmestyrings-controllere, som kan opfylde brugernes behov for et mere tidsbesparende system (Figur \ref{spg:priotid}). \\
Siri er en udbredt stemmestyringstekonologi, der benyttes i smartphonen iPhone. Apple er producenten af denne smartphone, som er en af de bedstsælgende mobiltelefoner i nyere tid (Europa).\cite{IDCIphone} Apple har indsamlet en fyldestgørende kundetilfreshedsrapport, som kan være relevant i forhold til at dokumentere den generelle kundeoplevelse af stemmestyringsteknologien Siri. Denne undersøgelse viser blandt andet:
\begin{itemize}
    \item 87\% af iPhone 4S brugere benytter sig af Siri mindst en gang om måneden. Af disse er 55\% fuldt tilfredse med Siri og 36\% delvist tilfredse.
    \item 51\% af iPhone brugere siger at det er ekstremt vigtigt, at der er en funktion der minder om Siri på deres næste telefon\cite{SiriUsage}.
\end{itemize}
Ud fra dette kan det konkluderes at stemmestyring er en teknologi, som i samspil med telefoner, har tilfredse brugere. Det kan ydermere konkluderes, at det er en teknologi, som brugerne ser i deres fremtidige elektroniske apparater. På baggrund af denne tilfredshed antages det, at en stemmestyrings implementationen af det automatiserede hjem vil være en populær løsning.