%Læsevejledning% %Jimmi og Martin%
%Sidst rettet: 15-12-2013, 17:57, Jimmi
{\bf Komposition} \\
Rapporten er kompositorisk opsat således at overemnet: ”Automatisér dit hjem” indledes som et vidtstærkt og udefinerbart begreb, hvilket gruppen indledningsvis ønsker at begrebsliggøre. I problemanalysen vil gruppen løbende foretage kritiske valg, som vil munde ud i en konkretiseret problemformulering. Denne problemformulering vil rekapitulere en konkret problemstilling, som rapporten vil tage udgangspunkt i det efterfølgende kapitel: Problemløsning. \\

{\bf Kilde henvisning} \\
Som kilde henvisninger anvendes Vanvouver-metoden. Dette bliver brugt ud fra hvert afsnit eller citat, som henviser til en specifik kilde i litteraturlisten vil der stå et tal indkapslet. \\
\textit{Eksempel: Påstand/citat\emph{[1]}}.\\

Hvis der i et afsnit er anvendt store mængder information fra en kilde, står kilden efter punktummet i afslutningen af afsnittet.\\
\textit{Eksempel: Afsnit.\emph{[1]}}\\

Litteraturlisten er kompositorisk opsat således, at kilderne er placeret efter den numeriske orden de er anvendt i teksten. Ud fra deres repræsentative nummerering er alle relevante informationer associeret. \\
\textit{Eksempel: Forfatter(e), Titel på artikel/afsnit, sider med relevant information, bogens titel, redaktør, forlagets hjemsted, forlag, udgivelses årstal.} \\

Hvis nogle af informationerne mangler, som f.eks. forfatterens navn, udelades disse informationer i kildebeskrivelsen. \clearpage

{\bf Figur henvisning} \\
I rapporten vil der løbende blive refereret til figurer eller illustrationer. I den kontekstuelle sammenhæng, hvor figurerne anvendes, vil dette være angivet på følgende måde: \textit{[Afsnit.nummer]} \\

\textit{Eksempel: 2. figur i 3. afsnit vil være angivet med følgende referencenummer: [3.2]}. \\

Under figuren vil figurenes referencenummer samt en dertilhørende figurbeskrivelse, som forklarer figurens relevans være påført den pågældende figur. \\ 
\textit{Eksempel: "Erklæring som en figur er kildemateriale til"}. Se figur \ref{fig:FigurEksempel}.\\
\figurw{Figurer/Figureksempel.png}{Eksempel på figur beskrivelse.}{FigurEksempel}{0.3}
%\figurw{Figurer/Sporgeskema/PrioBespar.png}{Hvor højt økonomisk besparelse prioriteres ved stemmestyring af hjem.}{priobespar}{1}
{\bf Fodnote henvisning}\\
Fodnoter benyttes til at inkludere yderligere informationer om et specifikt begreb, \textit{fagudtryk} eller forkortelse. Fodnoterne beskriver sjældent vitale informationer for forståelsen af det omtalte. En fodnote ses ved et lille tal som henviser til sit modstykke i bunden af siden, hvor den vedhæftede tekst er vist. \\

\textit{Eksempel: Fodnote\footnote{Eksempel på fodnote}}.\\

Første gang et specielt fagudtryk eller begreb bliver benyttet, introduceres det enten i en fodnote eller i den kontekstuelle sammenhæng. \\

{\bf Meta-tekst}\\
Hvert afsnit indledes med en meta-tekst som er anført i kursiv. \\

\textit{Eksempel: Dette afsnit indeholder...}. \\

{\bf Tabeller}\\
Løbende vil der blive fremvist tabeller, som vil blive anvendt konstruktivt i konteksten. Disse tabeller anvendes for at give læseren en bedre visualisering af tekstens indhold.

\begin{table}[h]
    \centering
    \begin{tabular}{ l | l }
        Overskrift & Overskrift \\
        \hline \hline
        A & B \\
    \end{tabular}
    \caption{\textit{Eksempel på en tabel. \tabelgroup}}
    \label{tab:abc}
\end{table}

{\bf Kodeeksempler}\\
Der vil i problemløsningens-afsnittet løbende blive refereret til kodeeksempler. 

\textit{Eksempel:}
\kodel{Kode/Code_example.c}{Beskrivelsen af den viste kode er placeret her}{Code_example}{1}{5}




