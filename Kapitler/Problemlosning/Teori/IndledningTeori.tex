%Skrevet af Jimmi
%Sidst rettet: 07-12-2013, 18:36, Jimmi
I følgende afsnit beskrives nogle af de handlingsprocessor, der skal til, for at stemmestyring kan omdannes fra et analog signal – i form af stemmen, til en programmerbar løsning. Teori-afsnittet er sat op på en måde, således at emnerne forekommer i den kronologisk rækkefølge; indledningsvis ønsker brugeren en kommando udført. Dette input omdannes til en tekststreg, som ledes videre til stavekontrol-afsnittet. Stavekontrollen er essentiel, for at rette eventuelle fejltolkningerne og misforståelser. Afslutningsvis beskriver afsnittet, tilstandsmaskiner, hvordan kontrollen skal foregå kontinuerligt, og i en stadiebaseret helhed.