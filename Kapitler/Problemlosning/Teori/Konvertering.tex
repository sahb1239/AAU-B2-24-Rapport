%Skrevet af Jimmi & Mads
%Sidst rettet: 29-11-2013, 16:07, Jimmi
%Sidst rettet: 9-12-2013, 17:14, Mads
Stemmestyrings er en yderst kompleks teknologi, som kræver en lang række operationer, før at inputtet – altså det sagte, kan oversættes til noget som den pågældende controller kan opfange, og videresende til PC-boksen. For at opnå et digitalt og forståeligt datasæt, skal følgende handlinger foretages i kronologisk rækkefølge, som illustreret på nedenstående Figur (\ref{fig:steps_jarvis}).  
\figurw{Figurer/steps_jarvis.png}{Ordet ”Jarvis” er blevet opfanget hos én af huset stemmestyringscontrollere. (Figuren er fremstillet af gruppen, og er blevet konstrueret på baggrund af Pracical Vibrations Analysis \cite{VoiceSteps}}{steps_jarvis}{1}
Stemmestyringsenhederne er blevet navngivet: ”Jarvis”. Dette er et unikt navn, som er assimileret med samtlige controllere i husstanden. Dette er primært for at forhindre systemet i at udføre handlinger, som ikke er målrettet det automatiserede system. Ovenstående Figur (\ref{fig:steps_jarvis}) skal illustrerer hvordan ordet: ”Jarvis” siges af brugeren (input), hvorefter støjende elementer frasorteres, inden konverteringsprocessen påbegyndes. \\\\
Dataen, som den pågældende controller opfanger, vil blive sendt videre til PC-boksen, hvor den videre behandlingsproces vil blive afviklet. Her undersøges der for, om ordet ”Jarvis” er godkendt – i så fald ændres systemet systemets tilstand (\ref{sec:tilstandsmaskine}) til at opfange alle efterkommende ord, idet at systemet nu ved, at brugeren ønsker at foretage en handling. De efterfølgende ord vil gennemgå den samme proces, for at afgøre om hvorvidt der skal sendes en datapakke\footnote{Mere info i afsnit \ref{sec:zensehome}} til det ønskede elektroniske apparat, eller ej. \\\\
Ved at optage og dissekere ordet: ”Jarvis”, kan der fremstilles en grafisk figur. Dette kan lade sig gøre, fordi stemmen består af en lang række forskellige vibrationer og frekvenser. Disse frekvenser kan dissekeres, ved at anvende en forbedret autokorrelations algoritme\footnote{Anvendes til at finde alle tonehøjde afsløringer}. Algoritmen bruges til at få et datasæt, som består af niveau\footnote{Niveauet måles i dB, sm er en enhed for lydstyrke}, frekvens\footnote{Frekvensens måles i Hz. For mere info se \ref{sec:frekvensens_struktur}} og strækning\footnote{Strækningen måles i tid, som afhængigt af datasæt, kan måles ned i tilnærmelsesvise milisekunder} \cite{Autocorrelation}. Idéen med algoritmen, er at fokusere på de fundamentale frekvenser – dette vil sige, at algoritmen fjerner alle negative værdier. Herefter genereres en kopi af optagelsen, hvilket fordobles i strækningstiden. Her anvendes korrektions-algoritmen. Dette gennemløb er for at fjerne alle højdepunkts beskæringer \cite{Autocorrelation}.  \\
Ud fra det genererede data, kan følgende graf fremstilles:
\figurw{Figurer/graph_jarvis.png}{Visualisere inputtet: 'Jarvis', opsat med niveauet som en funktion af tiden. (Figuren er fremstillet af gruppen, og er blevet konstrueret på baggrund af lydprogrammet: Audicity datasæt. jf. algoritmisk værktøj.)}{graf_jarvis}{1}
Figuren viser hvordan ordet ”Jarvis” udgør 6 forskellige spektre\footnote{Frekvensspektret er en målestok-enhed, der bruges til at finde frekvens udstrækning}, som danner rammerne for hvert ord i sætningen. Siges bogstavene separat, men indsættes i det samme tidsmæssige-spektrum, kan der ses en sammenligninghed mellem ordet og bogstaverne – de udgør tilnærmelsesvis det samme antal oktaver\footnote{Intervallet mellem en tonehøjde og en anden med en halv eller dobbelt dens frekvens}! Det der separere de forskellige bogstaver er nul-passage\footnote{Det punkt, hvor der ikke er nogen svingninger til stede}. Dette ses på figuren \ref{fig:graf_jarvis} som når niveauet er 0 \cite{VoiceProject}\footnote{Projektet er udarbejdet af David Wagner, som har en Ph. D. Dartmouth College - Concentration in Computer Science}. \clearpage
Zoomes der ind på ordets begyndelsesbogstav ”J” hvor bogstavet ”J” er sagt separat, kan sammenhængen ses bedre.
\figurw{Figurer/graf_jarvis_j.png}{Visualisere sammenhængen mellem ordet 'Jarvis' og begyndelsesbogstavet 'J'. (Figuren er fremstillet af gruppen, og er blevet konstrueret på baggrund af lydprogrammet: Audacity datasæt. jf. algoritmisk værktøj.)}{graf_jarvis_j}{0.5}
Som det ses på ovenstående figur, kan niveauet være vidt forskellige – til trods for, at der menes det samme. Det er et utal af forskellige parametre der spiller ind, som vil blive yderligere belyst i det kommende afsnit (afsnit 4.2.1.1). 
\clearpage
\subsubsection{Frekvensens struktur}
\label{sec:frekvensens_struktur}
Lydbølger består af en amplitude som afgør lydens niveau. For at afgøre lydbølgers toneleje, måler man på frekvens, som grundlæggende betyder, hvor tit bølgen har lavet en svingning. Figuren viser, hvordan antallet af svingninger er voksende, desto højere frekvenser er. Blå illustrere hvordan bølgelængden ville se ud, hvis svingningstiden var halveret. (Figur \ref{fig:curve_frequency})
\figurw{Figurer/frequency.png}{Illustrerer en svingning, fra bølgetop til bølgetop. (Figuren er fremstillet af gruppen)}{curve_frequency}{1}
Den gennemsnitlige kvinde har en lys og høj grundtonefrekvens på 180-200 Hz, hvor den mandlige stemme til modsætning er dybere, og har en omtrentlig frekvensområde på 100-130 Hz \cite{Stemme}. Dette skyldes stemmelæberne længde og tykkelse. Mænd har de længste og tykkeste, kvindernes er kortere og tyndere mens børns er kortest og tyndest. Det er dog først muligt at analysere på den gennemsnitlige grundtonefrekvens efter 20 års-alderen, idet at stemmelæberne skal være færdigudviklet.\\

Med denne teoretiske viden, vil det være muligt at konstruere systemet, således at antal svingninger over tid, vil være en indikator for den pågældende persons alder. Herved vil det være muligt, at opsætte vidt forskellige prioriteter for husets beboer. \\

Når inputtet er blevet konverteret til en tekststreng, skal tekststrengen kontrolleres for stavefejl, hvilket det næste afsnit (\ref{sec:stavekontrol}) vil komme nærmere ind på.