% Skrevet af Kim d. 03-12-2013
% Skrevet af Jimmi d. 05-12-2013 kl. 21.15
% Skrevet af Mads  d. 15-12-2013 kl. 14:30
% Rettet: Martin, 19:02 16-12-2013
\textit{I dette afsnit samles hele rapporten, for at undersøge, om de valg og afgrænsninger der blev truffet undervejs, var de korrekte. Ydermere diskuteres fejlkilder og hvordan dette har påvirket rapporten og dermed produktet.}
\\\\
{\bf Programmet}\\
Programmet har den funktion, at brugeren kan tænde for lyset ude i køkkenet, mens brugeren står på badeværelset. Derved er programmet designet på en sådan måde, at det skal respondere til forbrugeren at handlingen er udført, i dette tilfælde at lyset nu er tændt i køkkenet. Dette har en ulempe i den form, at programmet ville respondere hver eneste gang, der bliver givet en kommando. Dette vil være et potentielt irritationsmoment, da systemet vil respondere ved åbenlyse handlinger, som f.eks. hvis brugeren befandt sig i stuen og gav kommandoen: “Jarvis, tænd tv i stue”. Da ville systemet tænde tv´et og respondere: "Tv i stue er tændt". Dette er unødvendigt, da forbrugeren sidder i rummet, og derved selv kan se, at tv´et er tændt. Derved kan man overveje, om der skulle implementeres en form for positionering, således at systemet kan finde ud af, om forbrugeren er i rummet. Dette ville gøre at man kunne eliminere irritationsmomentet, men ligeledes give mulighed for, at forbrugeren simpelt kunne sige “Jarvis tænd lyset” uden at skulle angive en position. dette ville gøre systemet mere brugervenligt og ligeledes tidsbesparende. Måden man evt. kunne lave dette på ville være via triangulering\footnote{En teknik, hvorved man finder en position ved at måle på intensiteten af et input på tre forskellige måleinstrumenter.}, som gruppen ligeledes havde overvejet. Problemet i dette er dog, at man skal tage forbehold for hvor mikrofonerne, som bliver benyttet til triangulereringen, kan være placeret. Hvis mikrofonen er placeret bag en sofa, for eksempel, ville dette resultere i, at lyden opfanges forvrænget eller som værende længere væk, pga. lyds bevægelse igennem luft. Derfor har gruppen valgt ikke at udfører dette, da det anses for at være en udbyggelse, som er en mindre detalje uden betydning for programmets virkemåde.\\
Programmet har ligeledes ikke fået implementeret funktionen, at forbrugeren skulle kunne låse systemet overfor sine børn eller lignende. Dette ville være nyttigt i forhold til hvis forbrugeren har børn, hvor forælderen ønsker at slukke deres børns tv kl 20. Hvis børnene prøver at tænde det igen, skal forældrene kunne låse funktionen overfor børnene således, at de ikke kan tænde det. Problemet i dette er, at hvis forælderen skal kunne låse funktionen, ville den være låst indtil de låser den op igen. Hertil kunne man komme problemet til livs med indstillinger som gør, at funktionen åbner igen kl. 8.00 om morgenen, eller hvornår end forældrene specificerer det. Dette er dog valgt ikke implementeret, da tidsbegrænsningen (10 ECTS) ikke tillader det. Der er dog blevet implementeret et prioriteringssystem, som ikke tillader at specifikke brugere kan betjene visse enheder. \\

I programmets nuværende form er der adskillige ikke optimerede funktioner, som gør at programmet enten kører langsommere eller bruger mere plads, end er strengt nødvendigt. Dette sker fordi visse funktioner er konstrueret således, at de tog kortere tid for programmørerne at kode.
Et eksempel på dette, er hvis brugeren vil oprette en bruger i systemet. Her vil programmet få nogle input i form af et navn og en prioritet. Systemet åbner derefter filen “users.txt” og indlæser alt hvad der er i den. Derefter sletter systemet alt i filen og indsætter det som den har læst, samt den nye data, som brugeren har angivet.
Dette er upraktisk, da systemet bruger mere hukommelse på, at huske hvad der stod i filen, hvor man kunne diskutere, om programmet ikke skulle indsætte inputtet fra brugeren i slutningen af filen i stedet for. Dette har gruppen dog valgt ikke at gøre pga. tidsmæssige aspekter, og ligeledes i belysningen af, at det ikke har en indflydelse på programmets ydeevne. Grunden til at det skal ændres, selvom det ikke har en indflydelse på programmets ydeevne i gruppens version, er hvis der, for eksempel, er 50 brugere i systemet, ville det tage længere tid at indlæse, og derved ville det tidsbesparende element forsvinde.\\
En anden upraktisk ting er hvordan stavekontrollen indlæser kendte ord. Optimalt set skulle denne funktion også kunne indlæse ord fra controllere og scenarier. Dette gør den dog ikke, dette giver problemer efter oprettelse af en ny controller, fordi at stavekontrollen ikke kender det nye controllernavn og derfor ikke vil godtage input.
\\\\
{\bf Problemanalysen}\\
Problemanalysen blev opdelt i flere bestanddele, for at vi derved kunne foretage eksplicitte afgrænsningsvalg. Dette virkede intuitivt på daværende tidspunkt, idet at emnets mangfoldighed gjorde denne afgrænsning nødvendig, for den videre bearbejdning af rapporten. I et forsøg på at undersøge problemerne, der måtte være forbundet med et automatiseret hjem, blev der igangsat en dataindsamlingsproces. Denne proces blev foretaget i et forholdsvis tidligt stadie, som førte til meget abstrakte og ukonkrete spørgsmål. Den ideelle løsning kunne være et spørgeskema, som udelukkende fokuserede på problemerne forbundet med det automatiserede hjem, for derefter at behandle denne data. Denne data skulle bruges til at strukture et nyt spørgeskema, hvor fokuspunktet skulle irettesættes ud fra det fremfundne problem. Herved ville det indsamlede data være fuldstændig målspecifikt, i forhold til hvor det samfundsrelaterede problem forefindes, og hvad der kunne gøres for at udbedre det.
\\
Grundet tidspres (10 ECTS) kunne gruppemedlemmerne ikke nå, at foretage en ny markedsanalyse, hvilket førte til tynd argumentation og tvetydige konstateringer. Herunder blev der ikke stillet et spørgsmål om, i hvor høj grad, brugerne ønsker øget komfort i deres automatiserede hjem, hvilket var en klar fejltagelse. Dette førte istedet til, at gruppen foretog den antagelse, at hvis tidsforbruget kunne formindskes, ville dette ligeledes lede til øget komfort.
\\\\
{\bf Spørgeskema}\\
Spørgeskemaet blev publiceret igennem diverse fora, sociale medier, samt adspørgelser på gaden, hvorved gruppen opnåede et relativt højt besvarelsesantal (272 komplette). Besvarelserne blev rettet mod 2 målgrupper; dem der havde et automatiseret hjem (18 \%), og dem der ikke havde (82 \%). Størstedelen af disse besvarelser var sidstnævnte, som betød, at det var svært at påpege et specifikt problem ved automatiseringsløsningerne. Dette førte til, at gruppen arrangerede et interview med et automatiseringsfirma, Zensehome. Her blev der gennemført en lang række kvantitative spørgsmål, i forhåbninger om at kunne pointerer et specifikt problem, som den videre rapportudarbejdelse kunne tage udgangspunkt i. Det viste sig dog, at løsningen allerede var komplet, idet at systemet var blevet konstrueret med henblik på økonomiske besparelser, funktionelt brugerinteraktion, tidsbesparelse, samt øget komfort.\\
I anledning af, at der skulle tages en beslutning, om hvor rapportens fokus skulle ligges, blev de førnævnte data fra spørgeskemaet fundet frem. Her blev der draget tvetydige konstateringer, ud fra et relativt snævert og vidtrækkende udgangspunkt, om at manglende stemmestyring af automatiserede hjem kunne være et aktuelt problem.\\
I forbindelse med løsningsforslaget forsøgte gruppen, at forme nogle konkrete faciliteter. Disse faciliteter skulle anskueliggøre, hvordan de overordnede krav skulle opnås. Alle elementer i kravspecifikationen blev integreret i programmet, hvorved gruppen hypotetisk kunne konstatere, at ønsket om at mindske forbrugernes tidsforbrug, blev opnået. Det er dog besværligt, at påvise den direkte besparende effekt. Dette er også gældende for den personlige komfort, som kan tolkes meget subjektivt, og derved er utrolig svær at måle. Den tidsmæssige besparelse ville i høj grad afhænge af, hvordan husstandens beboere ville være villige til, at anvende stemmestyring som en aktiv funktion i deres automatiserede system.

%I dette afsnit diskuteres der på mulige fejlkilder i form af dataindsamlingen, samt argumentationen af problemafgrænsningen \\\\

% Skrevet af Kim d. 03-12-2013
%{\bf Problemafgrænsning}\\
%Ud fra den problemafgrænsning der er skrevet, kan man se hvordan der er blevet afgrænset ved hvert eneste emne i problemanalysen. Dette kan medføre at der bedre bliver skabt overblik over hvor og hvordan gruppen har valgt at afgrænse de problemstillinger der er dukket op. Ulempen der kan være ved denne måde at samle sin afgrænsning på, kunne være hvis der kom nye aspekter/problemstillinger ind på et senere tidspunkt. Dette kunne gøre det svært at få alle de forskellige aspekter/problemstillinger ind på samme tid, og derved kan der opstå et problem om hvorvidt der er misset et vigtigt emne eller ej. \\
%Gruppens argumentation igennem problemafgræsningen og derved også videre til problemformuleringen giver en god forståelse samt indblik i hvorfor det netop er der som der skal bruges kræfter på at komme frem til en effektiv løsning. Problemet som kan opstå kan dog igen være hvis der blev opdaget et nyt og væsentligt aspekt som ikke ville kunne undgås, der er argumentationen for valget for lukket til at der kan laves nye store ændringer. \\\\

% Skrevet af Kim d. 03-12-2013
%{\bf Spørgeskema}\\
%Hele spørgeskemaet som kan ses i bilaget, blev brugt af gruppen til at kunne indsamle en mængde data, som kunne give et indblik i hvordan befolknings synspunkter var på automatiserede hjem. Spørgeskemaet blev derefter sendt ud via diverse sociale medier som f.eks. facebook. Problemerne med at udstede et spørgeskema igennem sociale medier var at det gav nogle fejlkilder idet at spørgeskemaet generelt set henvendte sig til folk med eget hus og derved kendte til forskellige aspekter som en husejer som regel har. Ud fra de besvarelser som gruppen fik, viste det sig at kun 35\% var ejer af en bolig. Dette gav så en stor fejlkilde ved at de resterende 65\% af besvarelserne ikke havde så stor relevans fordi de ikke var ejere af en husstand, og derfor var der nogle af spørgsmålene som var komplet irrelevante for denne del af besvarelserne. \\
%Hvorfor er denne fejlkilde et stort problem i dataindsamlingen? \\
%Grunden til at problemet opstår er at de besvarelser som ikke har noget relevant sammenhæng praktisk kun kunne bruges til at få ideer til smarte løsninger, og ikke til om det kunne have indflydelse på husejerenes dagligdag. For at få mest muligt brugbar data ud af spørgeskemaet har gruppen valgt at bruge de besvarelser som teknisk set er irrelevant og brugt dem til at lave en kvantitativ statistik.